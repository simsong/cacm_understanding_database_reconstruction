\documentclass[jou,apacite]{apa6}
\usepackage{amsmath}

\title{Analysis of a Database Reconstruction Attack on Public Data}
\shorttitle{}

\twoauthors{Christian Martindale}{Simson Garfinkel}
\twoaffiliations{Center for Disclosure Avoidance, U.S. Census Bureau}{Center for Disclosure Avoidance, U.S. Census Bureau}

\abstract{In recent years, a certain type of database attack,
the database reconstruction attack, has become increasingly
more feasible due to rapid advances in attack algorithm
sophistication. We discuss how these attacks
function and demonstrate their effectiveness and efficiency. By applying database reconstruction techniques on statistical tables drawn from simulated data, we demonstrate that traditional disclosure avoidance techniques are often insufficient to fully protect respondent privacy, and also demonstrate that Laplace noise addition is much more effective at defending against database attacks by SAT solvers and linear optimizers.

\textbf{Keywords: database reconstruction attack, SAT solver, linear optimization, privacy,
disclosure avoidance}}


\begin{document}
\maketitle

\section{Problem Background}
Database intruders often seek to identify individuals from publicly
available data products such as tables and graphs.
Defending against such privacy breaches is
a high priority for statistical agencies, and so researchers
have developed a variety of techniques to prevent database users
from identifying any individual's personal information.

These techniques include:
\begin{enumerate}
  \item Cell Suppression, where the values of cells with small counts or few possible
        generating combinations are removed from the published table
  \item Row Swapping, where the data rows corresponding to individuals
        with similar values in certain key cells are switched
  \item Generalization, where numerical values are grouped into
        buckets corresponding to ranges, instead of giving the exact
        values for each entry in the table
  \item Top-and-bottom-coding, where the statistical groups at the high and low ends
        of the table are given without upper or lower bound (e.g.
        reporting the highest group for age as "80+" instead of
        "80-90" and "90-100")
\end{enumerate}

The goal of a database reconstruction attack is to
use public statistics released about confidential data to create a mathematical system of equations,
which then can be used to reconstruct the original (before the disclosure
avoidance techniques were applied) data set.
While the above techniques are not without merit, this paper will
demonstrate that they alone are insufficient in guarding data against
a modern DRA.

\section{The Database Reconstruction Attack: An Example}

\begin{table}[b]
\caption{Household 1 Responses (Not published)}\label{tab1}
\begin{tabular}{cccc}
\hline\\[-1.5ex]
Age & Sex & Race & Gen \\[0.5ex]
\hline\\[-1.5ex]
55 & Female & White & Grandparent\\[0.5ex]
24 & Male & White & Parent\\[0.5ex]
28 & Female & Black & Parent\\[0.5ex]
5 & Female & Black & Child\\[0.5ex]
8 & Male & White & Child\\[0.5ex]
\hline
\end{tabular}
\end{table}

\begin{table}[h]
\caption{Published Survey Results}\label{tab2}
\begin{tabular}{ccc}
\hline\\[-1.5ex]
Group & Number & Average Age \\[0.25ex]
\hline\\[-1.5ex]
Individuals & 9 & 27.44 \\[0.25ex]
Males & 4 & 31.75 \\[0.25ex]
Females & 5 & 24 \\[0.25ex]
Parents & 4 & 36.75 \\[0.25ex]
Grandparents & 1 & --- \\[0.25ex]
Children (0-12) & 2 & 6.5 \\[0.25ex]
Children (13-17) & 2 & 16 \\[0.25ex]
Whites & 7 & 30.57 \\[0.25ex]
Blacks & 2 & 24 \\[0.25ex]
Black parents & 1 & --- \\[0.25ex]
Black children & 1 & ---\\[0.25ex]
White parents & 3 & 39.66 \\[0.25ex]
White children & 3 & 13.33 \\[0.25ex]
Households & 2 & 27.85 \\[0.25ex]
\hline
\end{tabular}
\end{table}

This section shows how the DRA works by applying the technique to successively more complex scenarios. First, we consider a survey given to one household by a statistical agency. Table 1 shows the responses to the following survey:
\begin{verbatim}
Please provide the following for each member
of your family:
Age, Sex, Race, Generation.
\end{verbatim}

The statistical agency then publishes the data product in Table 2 based on the survey results. If information in a cell comes from only one individual, then that cell is suppressed to protect that individual's privacy, indicated in the table by an entry of '---'.

Retabulating these results in a numerical format with the keys in Table 3 gives the results in Table 4.

\begin{table}[t]
\caption{Tabulation Key}\label{tab3}
\begin{tabular}{c|c}
\hline\\[-1.5ex]
Key & Value \\[0.5ex]
\hline\\[-1.5ex]
Male & 0 \\[0.5ex]
Female & 1 \\[0.5ex]
White & 0 \\[0.5ex]
Black & 1 \\[0.5ex]
Child & 0 \\[0.5ex]
Parent & 1 \\[0.5ex]
Grandparent & 2 \\[0.5ex]
\hline
\end{tabular}
\end{table}

\begin{table}[t]
\caption{Encoded Published Survey Results}\label{tab4}
\begin{tabular}{c|c|c|c|c|c}
\hline\\[-1.5ex]
ID & Household & Age & Sex & Race & Generation \\[0.5ex]
\hline\\[-1.5ex]
1 & 1 & 24 & 0 & 0 & 1  \\[0.5ex]
2 & 1 & 28 & 1 & 1 & 1  \\[0.5ex]
3 & 1 & 55 & 1 & 0 & 2  \\[0.5ex]
4 & 1 & 5 & 1 & 1 & 0  \\[0.5ex]
5 & 1 & 8 & 0 & 0 & 0  \\[0.5ex]
6 & 2 & 50 & 0 & 0 & 1  \\[0.5ex]
7 & 2 & 45 & 1 & 0 & 0  \\[0.5ex]
8 & 2 & 15 & 1 & 0 & 0  \\[0.5ex]
9 & 2 & 17 & 1 & 0 & 0 \\[0.5ex]
\hline
\end{tabular}
\end{table}


The data in Table 4 is the 'ground truth' database that the attacker wishes to reconstruct. At the start of the attack, the attacker writes the table of 45 unknowns given in Table 5.

\begin{table}[t]
\caption{DRA Unknowns}\label{tab5}
\begin{tabular}{c|c|c|c|c|c}
\hline\\[-1.5ex]
ID & Household & Age & Sex & Race & Generation \\[0.5ex]
\hline\\[-1.5ex]
1 & H1 & A1 & S1 & R1 & G1  \\[0.5ex]
2 & H2 & A2 & S2 & R2 & G2  \\[0.5ex]
3 & H3 & A3 & S3 & R3 & G3  \\[0.5ex]
4 & H4 & A4 & S4 & R4 & G4  \\[0.5ex]
5 & H5 & A5 & S5 & R5 & G5  \\[0.5ex]
6 & H6 & A6 & S6 & R6 & G6  \\[0.5ex]
7 & H7 & A7 & S7 & R7 & G7  \\[0.5ex]
8 & H8 & A8 & S8 & R8 & G8  \\[0.5ex]
9 & H9 & A9 & S9 & R9 & G9  \\[0.5ex]
\hline
\end{tabular}
\end{table}

The reconstruction attack works by identifying constraint
equations given by the data table. Constraint equations are  mathematical formulae representing rules that the ground truth satisfies. For example, we have the
following statistic:
\begin{verbatim}
Individuals: 9, 27.44
\end{verbatim}

This can be written as a linear constraint equation: \footnote{Multiplication by 100 is used to avoid floating-point rounding errors.}
\[\frac{100(A1 + A2 + A3 + A4 + A5 + A6 + A7 + A8 + A9)}{9} = 2744\]

This equation is in 9 unknowns, and so is unsolvable alone.
However, the attacker can write more constraint equations.
Note that the Boolean expression $(A == B)$ is True if $A = B$ and False otherwise.

\begin{verbatim}
Grandparents: 1, XXX
\end{verbatim}


Becomes:
\begin{align*}
& (G1==2) + (G2==2) + (G3==2) + (G4==2) + \\
& (G5==2)+ (G6==2) + (G7==2) +\\
& (G8==2) + (G9==2) = 1
\end{align*}

\begin{verbatim}
Children 0-12: 2, 6.5
\end{verbatim}

Becomes the following two equations:

\begin{align*}
& (G1==0) + (G2==0) + (G3==0)+\\
& (G4==0)+  (G5==0) + (G6==0) +\\
& (G7==0) + (G8==0) + G9==0) = 2
\end{align*}
\begin{align*}
& (A1 * (G1==0) + A2 * (G2==0) + A3 * (G3==0) +\\
& A4 * (G4==0) +  A5 * (G5==0) + A6 * (G6==0) + \\
& A7 * (G7==0) + A8 * (G8==0) + A9 * (G9==0)) = 13
\end{align*}
Once the attacker has converted all the published statistics
into equations like the ones above, the equations together form a system of many equations in 45 unknowns. This system has one 'true' solution, equivalent to the
ground truth, and possibly many other 'false' solutions, which fit the constraints but are not equivalent to the ground truth.
The \textit{solution universe}, the set of all solutions that fit the encoded set of constraints, is originally potentially  quite large, containing many false solutions along with the true solution.
However, each time the agency publishes a new statistic, the attacker can generate new constraints and therefore narrow
down the set of possible solutions. Eventually, the attacker
will be able to narrow down the solution universe to just one solution, at which point the solution universe contains only
the ground truth. At this point, the attack has succeeded, and the statistical agency has failed to protect its respondents' privacy. Note here that the cell suppression disclosure avoidance technique does not prevent the attacker from performing the reconstruction attack, but rather simply gives one fewer constraint to work with per cell suppressed.
Furthermore, even if the number of constraints is insufficient to narrow down the solution universe to just one element, the attacker can still often identify personal data because of the high probability that all remaining solutions share values for a set of people. For example, if there are three remaining solutions in the solution universe, and all three contain a 75-year-old black male grandparent in household 1, then the attacker knows that this person is a real person in the database, even though the rest of the database may not have been completely reconstructed. Thus, the database has failed to protect user privacy even though the attack did not fully reconstruct the database.

\section{Methods of Attack}

The three most common and effective ways of solving a
system of constraint equations are:
\begin{enumerate}

\item \textbf{Brute Force} - trying every possible combination of solutions.\\

\item \textbf{SAT Solvers} - applying programs that quickly solve Boolean algebra problems.\\

\item \textbf{Optimization} -  using optimization software such as IBM's CPLEX to solve systems of linear constraints.

\end{enumerate}

For large data sets such as the U.S. Census, performing a brute-force attack is infeasible due to the fact that the runtime of brute force programs scales exponentially with the number of unknowns. However, optimizers and SAT solvers are both quite effective because they can reconstruct databases much more quickly than can a brute-force approach. Recent advances in SAT solver heuristics have enabled these programs to frequently solve systems with even millions of variables in a matter of minutes. Although public Census data tables are drawn from hundreds of millions of American individuals, the Census publishes billions of statistics from those individual responses, so there are still sufficient constraints to narrow down the solution universe enormously. In later examples, we will use a SAT solver to demonstrate how quickly these very complex programs can solve large systems of equations.

\section{Revisiting the Example}

In order to show how a SAT solver can be used to rapidly narrow down the solution universe for a data set, we will perform a mock DRA on a new set of responses to the survey given earlier.

The ground truth responses from two new households are given below in Table 6.
\begin{table}[b]
\caption{Survey Results}\label{tab6}
\begin{tabular}{c|c|c|c|c|c}
\hline\\[-1.5ex]
ID & Household & Age & Sex & Race & Generation \\[0.5ex]
\hline\\[-1.5ex]
1 & 1 & 80 & 1 & 1 & 2  \\[0.5ex]
2 & 1 & 40 & 0 & 0 & 1  \\[0.5ex]
3 & 1 & 70 & 1 & 0 & 2  \\[0.5ex]
4 & 1 & 30 & 1 & 1 & 1  \\[0.5ex]
5 & 1 & 90 & 0 & 0 & 2  \\[0.5ex]
6 & 2 & 10 & 0 & 1 & 0  \\[0.5ex]
7 & 2 & 10 & 0 & 1 & 0  \\[0.5ex]
8 & 2 & 10 & 1 & 0 & 0  \\[0.5ex]
9 & 2 & 40 & 1 & 0 & 1 \\[0.5ex]
10 & 2 & 20 & 0 & 1 & 1 \\[0.5ex]
\hline
\end{tabular}
\end{table}

The statistical agency publishes the statistics in Table 7 after processing the ground truth responses(SHOULD WE SHOW ALL CONSTRAINT STATS OR JUST A SAMPLE?).

\begin{table}[b]
\caption{Data Publication}\label{tab7}
\begin{tabular}{ccc}
\hline\\[-1.5ex]
Group & Number & Average Age \\[0.5ex]
\hline\\[-1.5ex]
Individuals & 10 & 40 \\[0.5ex]
Males & 5 & 34 \\[0.5ex]
Females & 5 & 46 \\[0.5ex]
Whites & 5 & 50 \\[0.5ex]
Blacks & 5 & 30 \\[0.5ex]
\hline\\[-1.5ex]
Children (0-12) & 3 & 10 \\[0.5ex]
White children & 1 & --- \\[0.5ex]
Black children & 2 & 10 \\[0.5ex]
\hline\\[-1.5ex]
Parents & 4 & 32.5 \\[0.5ex]
Male parents & 2 & 30 \\[0.5ex]
Female parents & 2 & 35 \\[0.5ex]
Parents over 40 & 0 & N/A \\[0.5ex]
\hline\\[-1.5ex]
Grandparents & 3 & 80 \\[0.5ex]
White grandparents & 2 & 80 \\[0.5ex]
Black grandparents & 1 & --- \\[0.5ex]
Male grandparents & 1 & --- \\[0.5ex]
Female grandparents & 2 & 75 \\[0.5ex]
\hline\\[-1.5ex]
Households & 2 & 40 \\[0.5ex]
Households with 3+ generations & 0 & N/A \\[0.5ex]
Single-parent households & 0 & N/A \\[0.5ex]
Childless households & 1 & --- \\[0.5ex]
Interracial married couples & 2 & 32.5 \\[0.5ex]
Same-sex married couples & 0 & N/A \\[0.5ex]
Households $\geq 40\% $ female & 2 & 32.5 \\[0.5ex]
Households $\geq 40\% $ black & 2 & 32.5 \\[0.5ex]

\hline
\end{tabular}
\end{table}

From this data, the attacker generates many constraint equations as demonstrated earlier,
and then inputs these constraints into a SAT solver. In this example, we use a free open-source SAT solver
called PicoSAT written in C. Output is given from the Java program Sugar, which wraps PicoSAT for usability improvements. Output is given in Table 8 and is tabulated for readability. Although the ID numbers are different, the ten people in the SAT solver output are identical to the ten people from the ground truth in Table 6, so the attack has succeeded in reconstructing the entire database and narrowing down the solution universe to size 1 in only 35.51 seconds.

\begin{table}[b]
\caption{Sugar Output}\label{tab8}
\begin{tabular}{c|c|c|c|c|c}
\hline\\[-1.5ex]
ID & Household & Age & Sex & Race & Generation \\[0.5ex]
\hline\\[-1.5ex]
1 & 2 & 10 & 1 & 0 & 0  \\[0.5ex]
2 & 2 & 20 & 0 & 1 & 1  \\[0.5ex]
3 & 1 & 30 & 1 & 1 & 1  \\[0.5ex]
4 & 2 & 10 & 0 & 1 & 0  \\[0.5ex]
5 & 2 & 40 & 1 & 0 & 1  \\[0.5ex]
6 & 2 & 10 & 0 & 1 & 0  \\[0.5ex]
7 & 1 & 40 & 0 & 0 & 1  \\[0.5ex]
8 & 1 & 80 & 1 & 1 & 2  \\[0.5ex]
9 & 1 & 90 & 0 & 0 & 2 \\[0.5ex]
10 & 1 & 70 & 1 & 0 & 2 \\[0.5ex]
\hline
\end{tabular}
\end{table}

\pagebreak

\section{Defending Against a DRA}
As demonstrated above, new techniques must be developed in
order to protect databases against reconstruction attacks, as traditional techniques such as cell suppression are insufficient defense. One of the simplest and most effective techniques in defending against the DRA is noise addition, where the publishing agency adds random values to data before publication in order to increase the size of the solution universe. For example, if the statistical agency takes a true value of age = 10, then randomly adds either -2, -1, 0, 1, or 2, and then publishes that number, there are now five elements in the solution universe for this value, where without noise addition there would be only one. Adding noise from a more complex distribution such as the Gaussian or Laplace distributions adds even more elements to the solution universe, since they are not discrete distributions like the example above. Noise addition allows an agency to publish more statistics without fearing that it has given attackers too many constraints to work with.

Especially noteworthy in the discussion of noise addition techniques is a special type of noise addition called the Laplace Mechanism CITATION HERE.
The mathematics behind this process are beyond the scope of this paper, but the important property of this distribution is that it is effective in creating noise that minimally impacts the utility of the statistics while still ensuring individual privacy. Informally, the proper addition of Laplace noise ensures that the choice of any individual to respond or not respond to a survey does not significantly impact the published data from the survey. Therefore, a person is free to respond truthfully to a survey covered by Laplace noise addition without worrying that his privacy will be compromised by his choice to respond.

\section{Problem Analysis}

With the dramatic improvement in the efficiency of optimizers and SAT solvers in the last decade, the database reconstruction attack is no longer a solely theoretical danger. The vast amount of data products the Census publishes each year gives a determined and informed attacker more than enough constraints to reconstruct some or all of a target database and breach the privacy of millions of people. Although the Census Bureau's current disclosure avoidance technique suite is often sufficient to defend against a cursory attempt at an identification attack, we have demonstrated through attack simulations that cell suppression and bucketing are not completely against the power of more sophisticated forms of attack.

The most effective technique for defending against a DRA is the addition of noise to results before publication. If the noise is generated by correctly applying the Laplace mechanism, it is possible to guarantee individuals' choice to respond to a survey cannot harm them while simultaneously preserving the accuracy of the data for benign use. In addition to continuing to use the traditional disclosure avoidance techniques, data agencies should make probabilistic noise addition a required part of the pre-publication disclosure review process in order to protect respondents from the database reconstruction attack.

\section{Appendices}

\subsection{Appendix 1: SAT and SAT Solvers}

The Boolean satisfiability problem, known as SAT, was the first problem to be proven NP-complete.CITATION COOK-LEVIN HERE This problem asks, for a given Boolean formula, whether replacing each variable with either True or False can make the formula evaluate to True.   A consequence of SAT being NP-complete is that any problem can be reduced in polynomial time (i.e., quickly) to an instance of the SAT problem. Once a problem has been reduced to an instance of the SAT problem, this SAT problem can be fed into a SAT solver to find possible solutions. Although the SAT problem is not solvable by algorithms in polynomial time, researchers have found many heuristic techniques for expediting this process. SAT solvers combine a variety of these techniques into one complex process, resulting in polynomial time solutions for the SAT problem in many cases.

Modern SAT solvers use a heuristic technique called Conflict-Driven Clause Learning, commonly referred to as CDCL. CDCL works by the following process:

\begin{enumerate}

\item Assign a value to a variable arbitrarily.
\item Use this assignment to determine values for the other variables in the formula (a process known as unit propagation).
\item If a conflict is found, backtrack to the clause that made the conflict occur and undo variable assignments made after that point.
\item Add the negation of the conflict-causing clause as a new clause to the master formula and resume from step 1.

\end{enumerate}

This process is much faster at solving SAT problems than previous processes used in SAT solvers because adding conflicts as new clauses has the potential to avoid wasteful 'repeated backtracks'. Additionally, CDCL and its predecessor algorithm, DPLL, are both provably complete algorithms and will always return either a solution or "Unsatisfiable" if given enough time and memory.

There are a wide variety of SAT solvers available to the public for minimal or no cost. Although a SAT solver requires the user to translate the problem into Boolean formulae before use,  programs such as Naoyuki Tamura's Sugar facilitate this process by translating user-input mathematical and English constraints into Boolean formulae automatically.

\subsection{Appendix 2: Scalability and The Zebra Problem}

To demonstrate the real capabilities of the SAT solver when faced with a very large constraint system, in contrast to our above relatively small examples, we will use a system with 155 variables and 1135 Boolean clauses. This system results from the CNF encoding of a famous problem known as the Zebra Problem CITATION HERE.
This problem is appropriate because its structure closely models the current Census data collection and publication process. The writer receives the survey answers from the respondents, redacts much of the information to avoid giving away PII, and then publishes a data product.

The size of this problem more closely approximates the number
of variables an attacker would need to generate constraints for when performing a DRA through queries on a massive public data set. It is too complex for the vast majority of humans to solve in any reasonable time.

This problem is stated as follows:

\begin{enumerate}
\item Five people have five different pets, smoke five different
       brands of cigarettes, have five different favorite drinks and
       live in five different houses.
\item The Englishman lives in the red house.
\item The Spaniard has a dog.
\item The Ukranian drinks tea.
\item The Norwegian lives in the leftmost house.
\item The Japanese smokes Parliaments.
\item The Norwegian lives next to the blue house.
\item Coffee is drunk in the green house.
\item The snail owner smokes Old Gold.
\item The inhabitant of the yellow house smokes Kools.
\item The Lucky Strikes smoker drinks orange juice.
\item Milk is drunk in the middle house.
\item The green house is immediately to the right of the ivory house.
\item The Chesterfield smoker lives next door to the fox owner.
\item The Kools smoker lives next door to where the horse is kept.
\end{enumerate}


\begin{itemize}
\item Given these conditions, determine who owns the zebra and who drinks water.
\item (The attacker is 'targeting' the water drinker and the zebra owner.)
\end{itemize}

Just as demonstrated in the previous example, this problem
can be encoded as a large set of Boolean equations. The problem
gives 15 explicit constraints which allow the attacker to generate more constraints, and each of the combinations of variables (five people, five houses, five drinks, etc.) can be represented as a Boolean variable, for example, "The Norwegian lives in the red house == False"

Once the attacker has generated every constraint possible from the given information, he can input the constraints into a SAT solver. In this example, we run the zebra problem through PycoSAT, a free Python wrapper of PicoSAT.

The results:

\begin{verbatim}
Solution:  [-1, -2, 3, -4, -5, -6,
-7, -8, -9, 10, 11, -12, -13, -14,
-15, -16, -17, -18, 19, -20, -21,
22, -23, -24, -25, -26, -27, -28,
-29, 30, 31, -32, -33, -34, -35,
-36, -37, -38, 39, -40, -41, 42,
-43, -44, -45, -46, -47, 48, -49,
-50, 51, -52, -53, -54, -55, -56,
57, -58, -59, -60, -61, -62, -63,
64, -65, -66, -67, 68, -69, -70,
-71, -72, -73, -74, 75, 76, -77,
-78, -79, -80, -81, 82, -83, -84,
-85, -86, -87, 88, -89, -90, -91,
-92, -93, 94, -95, -96, -97, -98,
-99, 100, -101, -102, -103, -104,
105, 106, -107, -108, -109, -110,
-111, -112, -113, 114, -115, -116,
-117, 118, -119, -120, -121, 122,
-123, -124, -125, -126, -127, 128,
-129, -130, -131, -132, -133, -134,
-135, -136, 137, -138, 139, -140,
-141, 142, -143, -144, -145, -146,
-147, 148, 149, -150, 151, -152,
-153, 154, -155]
Number of solutions found in 0.00046s :  1
\end{verbatim}

Each of the 155 numbers represents one of the variables encoded in the formula input to the SAT solver.
Positive integers correspond to True, while negative integers are False. For example, $118$ in the solution output means that the Boolean variable "The Spaniard drinks orange juice" is True.
PicoSAT is able to solve this very complex problem in a fraction of a second, and has reconstructed the entire database, thus obtaining the PII of all the residents of the village despite the fact that the vast majority of their information was redacted by the problem writer before release.

The success of this attack demonstrates the inefficacy of
cell suppression as a method for protecting PII against
a SAT solver-driven DRA. This problem
contained far more suppressed cells than unsuppressed ones, which is not realistic given that a real publishing agency would want to minimize the number of suppressed cells. Despite this over-suppression, the SAT solver was still able to easily solve for the values of all cells in the example.

\subsection{Appendix 3: Linear Optimizers}


\begin{thebibliography}{1}
\bibitem{diffprivacy} Cynthia Dwork, Aaron Roth.
\textit{The Algorithmic Foundations of Differential Privacy}, Foundations and Trends in Theoretical Computer Science, Volume 9, Nos. 3--4, pp. 211-407, 2014
\bibitem{sugar} Naoyuki Tamura, Akiko Taga, Satoshi Kitagawa, Mutsunori Banbara.
\textit{Compiling Finite Linear CSP into SAT}, Constraints, Volume 14, No. 2, pp.254--272, 2009.
\bibitem{zebra} Rina Dechter.
\textit{Enhancement Schemes for Constraint Processing: Backjumping, Learning, and Cutset Decomposition}, Artificial\\ Intelligence, No. 41, pp. 273-312, 1990
\bibitem{blank}
\end{thebibliography}


\end{document}
