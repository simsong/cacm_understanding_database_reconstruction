\documentclass[runningheads]{llncs}
\newcommand\NumConstraintLines{\xspace412\xspace}

\newcommand\NumSExpressions{\xspace60\xspace}

\newcommand\NumVariables{\xspace19,234\xspace}

\newcommand\NumClauses{\xspace1,113,650\xspace}

\newcommand\NumDIMACSLines{\xspace1,113,652\xspace}


\newif\ifanonymized
\anonymizedtrue
\newif\ifshortversion
\shortversionfalse
\newif\iflongversion
\longversiontrue

\anonymizedfalse
\shortversionfalse
\longversiontrue
% cmap has to be loaded before any font package (such as cfr-lm)
\usepackage{cmap}
\usepackage{amsmath}
\usepackage{amssymb}
\usepackage{siunitx}
%\usepackage[T1]{fontenc}
\usepackage{graphicx}
\usepackage[ngerman,english]{babel}
\usepackage{verbatim}
%better font, similar to the default springer font
%cfr-lm is preferred over lmodern. Reasoning at http://tex.stackexchange.com/a/247543/9075
\usepackage[%
rm={oldstyle=false,proportional=true},%
sf={oldstyle=false,proportional=true},%
tt={oldstyle=false,proportional=true,variable=true},%
qt=false%
]{cfr-lm}
%Sorts the citations in the brackets
%It also allows \cite{refa, refb}. Otherwise, the document does not compile.
%  Error message: "White space in argument"
\usepackage{cite}
%extended enumerate, such as \begin{compactenum}
\usepackage{paralist}
\usepackage{csquotes}  % for easy quotations: \enquote{text}
\usepackage{microtype} % enable margin kerning
\usepackage{url}       % tweak \url{...}
%improve wrapping of URLs - hint by http://tex.stackexchange.com/a/10419/9075
\makeatletter
\g@addto@macro{\UrlBreaks}{\UrlOrds}
\makeatother
%required for pdfcomment later
\usepackage{xcolor}
%enable nice comments
%this also loads hyperref
\usepackage{pdfcomment}
%enable hyperref without colors and without bookmarks
\hypersetup{hidelinks,
   colorlinks=true,
   allcolors=black,
   pdfstartview=Fit,
   breaklinks=true}
%enables correct jumping to figures when referencing
\usepackage[all]{hypcap}

\newcommand{\commentontext}[2]{\colorbox{yellow!60}{#1}\pdfcomment[color={0.234 0.867 0.211},hoffset=-6pt,voffset=10pt,opacity=0.5]{#2}}
\newcommand{\commentatside}[1]{\pdfcomment[color={0.045 0.278 0.643},icon=Note]{#1}}

%compatibality with packages todo, easy-todo, todonotes
\newcommand{\todo}[1]{\commentatside{#1}}
%compatiblity with package fixmetodonotes
\newcommand{\TODO}[1]{\commentatside{#1}}

%enable \cref{...} and \Cref{...} instead of \ref: Type of reference included in the link
\usepackage[capitalise,nameinlink]{cleveref}
%Nice formats for \cref
\crefname{section}{Sect.}{Sect.}
\Crefname{section}{Section}{Sections}

\usepackage{xspace}
%\newcommand{\eg}{e.\,g.\xspace}
%\newcommand{\ie}{i.\,e.\xspace}
\newcommand{\eg}{e.\,g.,\ }
\newcommand{\ie}{i.\,e.,\ }

%introduce \powerset - hint by http://matheplanet.com/matheplanet/nuke/html/viewtopic.php?topic=136492&post_id=997377
\DeclareFontFamily{U}{MnSymbolC}{}
\DeclareSymbolFont{MnSyC}{U}{MnSymbolC}{m}{n}
\DeclareFontShape{U}{MnSymbolC}{m}{n}{
    <-6>  MnSymbolC5
   <6-7>  MnSymbolC6
   <7-8>  MnSymbolC7
   <8-9>  MnSymbolC8
   <9-10> MnSymbolC9
  <10-12> MnSymbolC10
  <12->   MnSymbolC12%
}{}
\DeclareMathSymbol{\powerset}{\mathord}{MnSyC}{180}

% correct bad hyphenation here
\hyphenation{op-tical net-works semi-conduc-tor}
%%%%%%%%%%%%%%%%%%%%%%%%%%%%%%%%%%%%%%%%%%%%%%%%%%%%%%%%%%%%%%%%
%% END COPYING HERE
%%%%%%%%%%%%%%%%%%%%%%%%%%%%%%%%%%%%%%%%%%%%%%%%%%%%%%%%%%%%%%%%

% Two kinds of cite, for the long version and the short version
\begin{document}
\title{Understanding Database Reconstruction Attacks on Public Data}
\titlerunning{Database Reconstruction}
\author{Simson Garfinkel \and John Abowd \and Christian Martindale }
\institute{U.S. Census Bureau}

\maketitle
\begin{abstract}
Statistical agencies are mandated to publish summary statistics and
micro-data while not providing data users with the ability to derive
specific attributes for particular persons or
establishments. 
Traditionally, these privacy guarantees are assured through the
use of \emph{Statistical Disclosure Limitation} (SDL)
techniques. These techniques are not sufficient to
prevent a database reconstruction attack of the sort anticipated by
Dinur and Nissim (2003). To illustrate the problem, this paper
presents a database reconstruction attack on a hypothetical block for
which statistics have been published by an official statistics
agency. We then show how the use of formally private noise reduces the
privacy risk of the database reconstruction. The paper concludes with
a discussion of the implications for the 2020 Census of Population and
Housing. 
\end{abstract}

\begin{keywords}
database reconstruction attack, SAT solver, 
\ifshortversion\else privacy,\fi 
disclosure avoidance
\end{keywords}

\section{Introduction}
In 2020 the Census Bureau will conduct the constitutionally mandated
decennial Census of Population and Housing, with the goal of counting
every person once, and only once, and in the correct place, and to
fulfill the Constitutional requirement to apportion the seats in the
U.S. House of Representatives among the states according to their
respective numbers.

Beyond the Constitutionally mandated role of the decennial census, the
US Congress has mandated many other purposes for the data. For
example, the U.S. Department of Justice uses block-by-block counts by
race for enforcing the Voting Rights Act. More generally, the results
of the decennial census, combined with other data, are used to help
distribute more than \$675 billion in federal funds to states and
local organizations.

In addition to collecting and distributing data on the American
people, the Census Bureau is charged with protecting the Privacy confidentiality of
survey responses. Specifically, all Census publications must uphold the
confidentiality standard specified by Title 13 of the U.S. Code, which
states, in part, that Bureau publications are prohibited from
identifying ``the data furnished by any particular
establishment or individual.''[Title 13, Section 9] This section
prohibits the Bureau from publishing respondent names, addresses, or any other
information that might identify a specific person or establishment.

Upholding this confidentiality requirement frequently poses a
challenge, because many statistical can inadvertently provide
information in a way that can be attributed to a particular
entity. For example, a survey of salaries in a region might report the
average salary earned by people of different occupations. However, if
there is only one bricklayer, than reporting the average salary of a
bricklayer will allow anyone who sees to statistical product to infer
that person's salary. If there are two bricklayers in the region,
reporting the average salary will allow each bricklayer to infer the
other's salary. Both of these cases would be clear violations of Title
13. The Bureau has traditionally used cell suppression to protect
privacy in situations such as this: the cells of statistical tables
that result from \emph{small counts} are suppressed and not
reported. If totals are reported, cell suppression requires that
additional \emph{complementary} cells be suppressed so that the values
of the suppressed cells cannot be worked out with simple arithmetic. 

In 2003, Dinur and Nissim showed that simple cell
suppression is not sufficient to protect the underlying confidential
data collected and used by a statistical agency to produce official
statistics\cite{DinurNissim2003}. To the contrary, they showed that
once an agency publishes more than a critical number of statistics,
the underlying confidential data can be \emph{reconstructed} by simply
finding a consistent set of microdata that, when tabulated, produce
the official statistics. One of the big surprises of the 2003 paper is
that the number of tabulations required to enable a database reconstruction
attack (DRA) is far fewer than might be intuitively expected: this is
because of the internal constraints and consistency requirements
in the published data. 

While it is mathematically impossible prevent reconstruction of the
underlying data, a data publisher can add noise to the published
results so that the reconstructed data will not reveal the actual
confidential responses that was used to create the published
tables. Clearly, as more noise that is added, the respondents will
enjoy greater privacy protection, but the data publications will have
less resulting accuracy. 

So how much noise needs to be added to protect privacy? Three years
later, Dwork, McSherry, Nissim and Smith answered that question. In
their paper ``Calibrating Noise to Sensitivity in Private Data
Analysis,''\cite{Dwork:2006:CNS:2180286.2180305} the researchers
introduced the concept of \emph{differential privacy}. The paper
provides a mathematical definition of the privacy loss that
persons suffer as a result of a data publication, and proposes a
mechanism for determining how much noise needs to be added for any
given level of privacy protection.

The 2020 census is expected to count roughly 320 million people living
on roughly 8.5 million inhabited blocks, with some blocks having as
few as a single person and other blocks having thousands. With this
level of scale and diversity, it is difficult to visualize how such a
data release might be susceptible to a database reconstruction
attack. Nevertheless, with recent improvements in both computing power
and big data applications, such reconstructions now pose a significant
risk to the confidentiality of microdata that underlies unprotected
statistical tables.

To help understand the urgency of adopting formal privacy methods,
this paper presents a database reconstruction attack on a much smaller
statistical publication: a demographic publication of a hypothetical
block containing seven people distributed over two households. We show
that even a relatively small number of constraints results in an exact
solution for many of the blocks' inhabitants. After we show the
successful reconstruction attack, we show how differential privacy can
protect the published data. Finally, we discuss implications for the
decennial census.

\section{The Database Reconstruction Attack: An Example}

To present the attack, we consider a hypothetical
census of a fictional geographic frame (for example, a suburban block)
conducted by a fictional statistical
agency. For every household, the agency collects each resident's age,
sex, race, and their relation in the household. To simplify the example,
the fictional world has only two races, black and white, and two
sexes, female and male. The statistical agency
is prohibited from publishing the raw microdata, and instead publishes
a tabular report (Table~\ref{fictional}). 

Notice that a substantial amount of information in
Table~\ref{fictional} has been suppressed (censored). In this case,
statistics resulting from one or two people are supported, but
statistics from three people are provided. This suppression rule is
sometimes called ``the rule of three.''

\newcommand{\cens}{\multicolumn{1}{c|}{\rule{6mm}{3mm}}}
\begin{table}
\begin{center}
\begin{tabular}{l|l|c|c|c|}
     &                           &       & \multicolumn{2}{|c|}{Age} \\
Item & Group                     & Count & Median & Average \\
\hline
  1A & total population          & 7     &  30    & 38 \\
\hline
  2A & female                    & 4     &  30    & 33.5 \\
  2B & male                      & 3     &  30    & 44 \\
  2C & Black or African American & 4     &  51    & 48.5 \\
  2D & White                     & 3     &  24    & 24 \\
\hline
  3A & single adults             & \cens & \cens  & \cens \\
  3B & married adults            & 4     & 51     & 54 \\
\hline
  4A & Black or African American female              & 3     & 36     & 36.7 \\
  4A & Black or African American male                & \cens & \cens  & \cens \\
  4B & White male                & \cens & \cens  & \cens \\
  4B & White female              & \cens & \cens  & \cens \\
\hline
  5A & persons under 5 years     & \cens & \cens  & \cens \\
  5B & persons under 18 years    & \cens & \cens  & \cens \\
  5C & persons 64 years or over  & \cens & \cens  & \cens \\
\hline
\multicolumn{5}{l}{Note: Married persons must be 15 or over}
\end{tabular}
\caption{Fictional statistical data for a fictional block published by
  a fictional statistics agency. Item numbers are for identification
  purpose only.\label{fictional}}
\end{center}
\end{table}

\begin{center}\begin{table}\begin{tabular}{rrr}
a &  b &  c\\ 
\hline
 1 &  30 &  101\\ 
2 &  30 &  100\\ 
3 &  30 &  99\\ 
4 &  30 &  98\\ 
5 &  30 &  97\\ 
6 &  30 &  96\\ 
7 &  30 &  95\\ 
8 &  30 &  94\\ 
9 &  30 &  93\\ 
10 &  30 &  92\\ 
11 &  30 &  91\\ 
12 &  30 &  90\\ 
13 &  30 &  89\\ 
14 &  30 &  88\\ 
15 &  30 &  87\\ 
16 &  30 &  86\\ 
17 &  30 &  85\\ 
18 &  30 &  84\\ 
19 &  30 &  83\\ 
20 &  30 &  82\\ 
21 &  30 &  81\\ 
22 &  30 &  80\\ 
23 &  30 &  79\\ 
24 &  30 &  78\\ 
25 &  30 &  77\\ 
26 &  30 &  76\\ 
27 &  30 &  75\\ 
28 &  30 &  74\\ 
29 &  30 &  73\\ 
30 &  30 &  72\\ 
\end{tabular}
\caption{The 30 possible ages for which the median is 30 and the mean is 44}\label{medians}\end{table}\end{center}\newcommand\mymedian{30}\newcommand\mymean{44}\newcommand\myconsidered{253460}\newcommand\mycount{30}

To perform the database reconstruction attack, we view the attributes
of the persons living on the block as a collection of 
free variables. We then extract from the published table a set of
constraints. The database reconstruct attack merely finds a set of
attributes that are consistent with the constraints. If statistics are
highly constrained, the only a single reconstruction will be
possible, and that reconstruction should be the same as the underlying
microdata used to create original table.

For example, statistic 2B states that there are 3 males living in the
geography.  Because age is typically reported to the nearest year, and
age must necessarily be $\ge 0$ and $\le 115$, and as a result there
are only a finite number of possible combinations, specifically:

\[ \binom{116}{3}=\frac{116 \times 115 \times 114}{3 \times 2 \times
  1} = 253,460 \]

Within these sets, there are \mycount{} combinations that satisfy the
constraint of having a median of \mymedian{} and a mean of \mymean{}
(from Table~\ref{fictional}). This means that there are
\mycount{} different databases that satisfy statistic 2B.

To mount a full reconstruction attack, an attacker extracts all of the
constraints specified by the published statistics and then creates a
single mathematical model that reflects all of the
constraints. Although this might seem difficult to do in practice,
computational advances within the past decade have made such
reconstruction attacks relatively straightforward.

In this example, the census has collected four attributes for each
individual: sex, age, race, and marital status. With seven individuals,
there are $7\times 4 = 28$ variables that need to be solved. These
variables appear in Table~\ref{variables}

\begin{center}
\begin{table}
\begin{tabular}{l|cccc}
       &     &     &      & Marital  \\
Person & Age & Sex & Race & Status   \\
\hline                             
1      & S1  & S1  & R1   & M1       \\
2      & S2  & S2  & R2   & M2       \\
3      & S3  & S3  & R3   & M3       \\
4      & S4  & S4  & R4   & M4       \\
5      & S5  & S5  & R5   & M5       \\
6      & S6  & S6  & R6   & M6       \\
7      & S7  & S7  & R7   & M7       \\
\hline
\\
\multicolumn{1}{l}{Key:}\\
\hline
female &     &  0  & \\
male   &     &     & \\
\hline
black  &     &     &  0   & \\
white  &     &     &  1   & \\
\hline
single &     &     &      &   0\\
married&     &     &      &   1\\
\hline
\end{tabular}
\caption{The variables associated with the database reconstruction
  attack. The coding for categorical attributes is presented in the key.}\label{variables}
\end{table}
\end{center}

To continue with our example, statistic 2B contains three
constraints. In our example, female persons
are represented by 0
and male persons by 1, so the constraint that there are four female persons can be represented as:

\begin{equation}\label{eq1}
S1 + S2 + S3 + S3 + S5 + S6 + S7 = 3
\end{equation}

Alternatively, we could create subtract each variable $S_n$ from one
to create a variable that is 1 for male and 0 for female. With this
notation, a constraint for the number of male persons could be written as:

\begin{equation}
(1-S1) + (1-S2) + (1-S3) + (1-S3) + (1-S5) + (1-S6) + (1-S7) = 3
\end{equation}

From here, we can write a constraint for the average age of the male persons:

\begin{equation}
\frac{
  \begin{split}
  A1 \times (1-S1) + A2 \times (1-S2) + A3 \times (1-S3) + \\
   A4 \times (1-S4) + A5 \times (1-S5) + A6 \times (1-S6) + A7 \times (1-S7)
  \end{split}
}{3} = 44
\end{equation}

More complicated constraints can be represented using a functional
programming language. 


Once all the published statistics
are translated into constraints, the equations together form a
system of many 21 variables with 28 unknowns. However, there are
other, less obvious constraints. One constraint is that the variables
are all integers. Another constraint, imposed by the statistical
agency, is that there is a minimum age difference between a child and
their parent.

There exists a \textit{solution universe} of all the possible solutions to
this set of simultaneous equations. If there is a single possible
solution, then the published statistics completely reveal the
underlying confidential data---the ground truth. However, if
there is more than one possible solution, then one of those solutions
is correct, and the others are not. If the equations have no solution,
then the set of published statistics is inconsistent.

\section{Performing the Attack}
A straightforward but wildly inefficient approach to solving this
system of simultaneous equations is to perform a brute force
search---that is, to try all possible values of all possible variables
and record the ones that work.

Ordinarily a brute force search on even this toy problem would be
unreasonable: there are 7 age variables that can range from 0 to 115
and 21 binary variables (sex, race race and marital status). All
together, there are $116^{10} \times 2^{21} \approx 9.25
\times 10^{26}$ possible combinations. Even if we could try a billion
a second, trying all possible combinations would take 90,000 trillion
years. 

But there is no need to try all possible combinations: the highly
constrained system can frequently be solved automatically using a
program called a Satisfiability (SAT) solver. 

Those not familiar with modern SAT solvers may be somewhat
incredulous at trying to  solve
a problem with a work factor of roughly $10^{34}\approx2^{113}$.
However, ``The past few years have seen an enormous progress in the performance
of Boolean satisfiability (SAT) solvers. Despite the worst-case
exponential run time of all known algorithms, satisfiability solvers
are increasingly leaving their mark as a general-purpose tool in areas
as diverse as software and hardware verification,
automatic test pattern generation, planning,
scheduling, and even challenging problems from algebra. Annual SAT
competitions have led to the development of dozens
of clever implementations of such solvers, an exploration of many new
techniques, and the creation of an extensive suite of real-world
instances as well as challenging hand-crafted benchmark
problems. Modern SAT solvers provide a ``black-box'' procedure that
can often solve hard structured problems with over a million variables and
several million constraints.''\cite[references omitted]{Gomes200889}.

Many SAT solvers take their input in the so-called DIMACS file format,
which specifies a single equation in conjunctive normal form (CNF). To use these solvers, a
preprocessor transforms the equations into CNF, runs the SAT solver,
and then translates the results back. The program we use below is
called Sugar\cite{sugar}, which takes as input a a series of
\textit{s-expressions}\cite{McCarthy:1960:RFS:367177.367199}. For
example, the constraint in Equation~\ref{eq1} can be encoded as the following
s-expression:
\begin{equation}
\texttt{(= (+ S1 S2 S3 S4 S5 S6 S7) 3)}
\end{equation}

Likewise, constraint for mean age can be encoded as:

\begin{equation}
\texttt{(= (+ A1 A2 A3 A4 A5 A6 A7)\\ (* 7 38))}
\end{equation}


Notice that we avoid round-off error by not encoding the division
operator. This is vital, as the SAT encoding used by Sugar only allows
for integer arithmetic. 

Our source input file consists of \NumSExpressions s-expressions in \NumConstraintLines{} lines (minus comments).\footnote{The entire file can be downloaded from our website.}
This converts into
\NumVariables Boolean variables and \NumClauses CNF clauses. Using the
open source SAT solver PicoSAT\cite{Biere_picosatessentials}, we are able
to solve this system in typically less than 2 seconds on a 2013 MacBook Pro with a 2.8GHz Intel
i7 processor and 16GiB of RAM (although the program was not limited by
RAM). Output is given in Table~\ref{sugarbig} and is tabulated for readability.

Examining the output, we see that 5 of the 7 fictional respondents are
exactly solved. Because PicoSAT stops at the first satisfying
solution, other runs of the SAT solver might solve correctly for
more or fewer. 

%At past international SAT solver competitions, 
SAT solvers such as PicoSAT can solve problems
with tens of millions of variables in less than 20 minutes
\cite{satcomp}, making it likely that today's SAT solvers could solve
realistic database reconstruction problems based on data being
published today by official statistics agencies.


\begin{table*}
\begin{minipage}[t]{.33\linewidth}
~\\
\begin{tabular}{ccccc}
    &     &      & Marital \\
Age & Sex & Race & Status \\
 8  &  F  &   B  &   S  \\
18  &  M  &   W  &   S  \\
24  &  F  &   W  &   S  \\
30  &  M  &   W  &   M  \\
36  &  F  &   B  &   M  \\
66  &  F  &   B  &   M  \\
84  &  M  &   B  &   M  \\
\hline
\end{tabular}

\caption{Ground Truth Data}
\end{minipage}
\hfill
\begin{minipage}[t]{.33\linewidth}
~\\
% The variables in this tempalte table are replaced by the
% run_reconstruction.py program
% 
\begin{tabular}{ccccc}
     &     &      & Marital \\
 Age & Sex & Race & Status \\
 8  & F  & B   & S  \\
 24  & F  & W   & S  \\
 36  & F  & B   & M  \\
 66  & F  & B   & M  \\
 84  & M  & B   & M  \\
 18  & M  & W   & S  \\
 30  & M  & W   & M  \\
\hline
\end{tabular}

\caption{Solved with all statics}
\end{minipage}
\hfill
\begin{minipage}[t]{.33\linewidth}
~\\
\begin{tabular}{cccccc}
   &     &     &      & Marital \\
ID & Sex & Age & Race & Status \\
\hline
1  & M  & 24  & W   & S  \\
2  & F  & 25  & B & S  \\
3  & F  & 47  & W & M  \\
4  & M  & 78  & B & M  \\
5  & F  & 1  & W & S  \\
6  & F  & 61  & B & M  \\
7  & M  & 30  & B & M  \\
\hline
\end{tabular}

\caption{Solved without the use of statistic 4A}
\end{minipage}
\hfill
\end{table*}

\section{Defending Against a DRA}\label{solution}
There are three approaches for defending against a database reconstruction: publish less
statistical data, and apply noise (random changes) to the
statistical data being tabulated, or apply noise to the results after
the tabulation. We consider them in order below.

Although it might seem that publishing less statistical data is a
reasonable defense against the DRA, this choice may severely limit the number
of tabulations that can be published. A
related problem is that it may be computationally infeasible to
determine when a reconstruction is possible from limited publications.

A second approach is to apply noise to the data before
tabulation. For example, each respondent's age might be randomly
altered by a small amount, perhaps $-2 \ldots 2$; 
the statistical agency then tabulates the values. 
In this scenario, if the attacker performs a DRA, the result is the altered dataset, and not the underlying true values. Thus, if two responses
are processed with noise of this sort, there are now $5^2 = 25$
possible solutions. If noise infusion is applied to all 7 age
variables, there are $5^{7}=78,125$ possible
solutions. Database reconstruction may still be possible, but there is no
way to know which reconstructed databse is correct. 

A third approach is to use output noise infusion. Instead of reporting
precise answers for each of the quoted statics, each is reported with
the addition of noise.  With output noise infusion, the attacker must
revise each constraint from an equality to a bounded range, resulting
again in an explosion of possible solutions that is exponential in the
number of constraints. Exact solutions can still be found, but there
is no way for an attacker to determine which are correct.  Like input
noise infusion, output noise infusion does not prevent an attacker
from reconstructing the database, but it prevents the attacker from
knowing if the reconstruction is correct.

\textbf{How much noise should be added? --- have a brief intro to
  differential privacy here, and then the big finish.}

\section{Related Work}

Dinur and Nissim\cite{noise} showed that the underlying
confidential data of a statistical database can be reconstructed with
a surprisingly small number of queries. In practice, most statistical
agencies perform these queries themselves when they release
statistical tables. Thus, Dinur and Nissim's primary finding
is: a statistical agency that publishes too many accurate statistical
tables drawn from the same confidential dataset risks inadvertently
compromising that dataset unless it takes specific protective measures.

Statistical tables create the possibility of database reconstruction
because they form a set of constraints for which there is ultimately
only one exact solution. Restricting
the number or specific types of queries---for example, by suppressing
results from a small number of respondents---is often insufficient to prevent access
to indirectly identifying information, because the system's refusal to
answer a ``dangerous'' query itself provides the attacker with information. 
Dinur and Nissim found that, if a database is modeled as a string of $n$ bits,
then at least $\sqrt{n}$ bits must be modified by adding noise to
protect persons from being identified.

Kasiviswanathan, Rudelson, and Smith\cite{Kasiviswanathan:2013:PLR:2627817.2627919} introduced
the concept of the linear reconstruction attack, which underlies the  DRA. The key concept is that,
given nonsensitive data such as zip code or gender, an attacker
can construct a matrix of linear equalities that can often be solved
in polynomial time. \iflongversion The paper also analyzes a common reconstruction
technique known as least squares decoding, where the attacker sets up
a goal function to minimize the square of the distance between two
databases in order to reconstruct the original database.\fi

\iflongversion
Brown and Heathers\cite{doi:10.1177/1948550616673876} developed the
granularity-related inconsistency of means (GRIM) test in response to
observed inconsistencies in published data from psychological
journals. This test is centered around the premise that, for
statistics drawn from integer data, only certain means are
possible. The GRIM test determines whether reported means could
possibly have come from data sets with a certain size, granularity,
and group number. In surveying 71 published articles, the authors
found 36 papers with one inconsistency and 16 with two or more
inconsistencies. Although this test was intended to detect possible
errors or mean falsification in published articles, the concept of
drawing inferential conclusions about a data set based only on 
published statistics is a key concept behind the DRA.
\fi



\section{Conclusion}

With the dramatic improvement in the efficiency of SAT solvers in the
last decade, database reconstruction attacks are no longer just a 
theoretical danger. The vast quantity of data products published by
statistical agencies each year may give a determined attacker 
more than enough constraints to reconstruct some or all of a target database and
breach the privacy of millions of people. Although a suite of traditional
disclosure avoidance technique is often sufficient to defend against a cursory attack, cell suppression and generalization are not secure against more sophisticated
forms of attack. Agencies may also be using input and output noise infusion, but 
current systems may not be noisy enough to materially control the DRA risk.
Formal privacy methods, especially differential privacy, were specifically
designed to to control this risk and, as both of our noise-infusion examples
illustrate, they do so by systematically expanding the universe of solutions
to the DBA constraints. In this expanded universe, the real confidential data
are but a single solution, and no evidence in the published data can improve 
an attacker's guess about which solution is the correct one.

\iflongversion \section{References}\fi

\bibliographystyle{splncs03}
\bibliography{white_paper}
\iflongversion
\newpage
\emph{These appendices will not be part of the FC2018 submission, but
  would be part of a technical report.}

\section{Appendices}

\subsection{SAT and SAT Solvers}

The Boolean satisfiability problem (SAT) was the first
problem to be proven NP-complete\cite{cooklevin}. This problem asks,
for a given Boolean formula, whether replacing each variable with
either True or False can make the formula evaluate to True.  A
consequence of SAT being NP-complete is that any problem can be
reduced in polynomial time (i.e., quickly) to an instance of the SAT
problem. Once a problem has been reduced to an instance of the SAT
problem, this SAT problem can be fed into a SAT solver to find
possible solutions. Although the SAT problem is not solvable by
algorithms in polynomial time, researchers have found many heuristic
techniques for expediting this process. SAT solvers combine a variety
of these techniques into one complex process, resulting in polynomial
time solutions for the SAT problem in many cases.

Modern SAT solvers use a heuristic technique called Conflict-Driven
Clause Learning, commonly referred to as CDCL\cite{cdcl}. Briefly, CDCL algorithm
is:

\begin{enumerate}

\item Assign a value to a variable arbitrarily.
\item Use this assignment to determine values for the other variables
  in the formula (a process known as unit propagation).
\item If a conflict is found, backtrack to the clause that made the
  conflict occur and undo variable assignments made after that point.
\item Add the negation of the conflict-causing clause as a new clause
  to the master formula and resume from step 1.

\end{enumerate}

This process is much faster at solving SAT problems than previous
processes used in SAT solvers because adding conflicts as new clauses
has the potential to avoid wasteful ``repeated backtracks.''
Additionally, CDCL and its predecessor algorithm, DPLL, are both
provably complete algorithms and will always return either a solution
or ``Unsatisfiable'' if given enough time and memory.

There are a wide variety of SAT solvers available to the public for
minimal or no cost. Although a SAT solver requires the user to
translate the problem into Boolean formulae before use, programs such
as Naoyuki Tamura's Sugar facilitate this process by translating
user-input mathematical and English constraints into Boolean formulae
automatically.

\subsection{Sugar Input}

Sugar input is given in a standard Constraint Satisfaction Problem
(CSP) file format. A constraint must be given on a single line of the
file, but here we separate most constraints into multiple lines for
readability. Constraint equations are separated by comments describing
what statistics they encode.

Input for the model in this paper is as follows:

\verbatiminput{constraints.csp}
\fi

\end{document}

% LocalWords:  microdata equalities satisfiability
