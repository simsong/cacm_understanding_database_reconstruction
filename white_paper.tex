\newif\ifanonymized
\anonymizedtrue
%\anonymizedfalse
\documentclass[runningheads]{llncs}
%cmap has to be loaded before any font package (such as cfr-lm)
\usepackage{cmap}
\usepackage[T1]{fontenc}
\usepackage{graphicx}
\usepackage[ngerman,english]{babel}
%better font, similar to the default springer font
%cfr-lm is preferred over lmodern. Reasoning at http://tex.stackexchange.com/a/247543/9075
\usepackage[%
rm={oldstyle=false,proportional=true},%
sf={oldstyle=false,proportional=true},%
tt={oldstyle=false,proportional=true,variable=true},%
qt=false%
]{cfr-lm}
%Sorts the citations in the brackets
%It also allows \cite{refa, refb}. Otherwise, the document does not compile.
%  Error message: "White space in argument"
\usepackage{cite}
%extended enumerate, such as \begin{compactenum}
\usepackage{paralist}
%for easy quotations: \enquote{text}
\usepackage{csquotes}

%enable margin kerning
\usepackage{microtype}

%tweak \url{...}
\usepackage{url}
%improve wrapping of URLs - hint by http://tex.stackexchange.com/a/10419/9075
\makeatletter
\g@addto@macro{\UrlBreaks}{\UrlOrds}
\makeatother
%required for pdfcomment later
\usepackage{xcolor}
%enable nice comments
%this also loads hyperref
\usepackage{pdfcomment}
%enable hyperref without colors and without bookmarks
\hypersetup{hidelinks,
   colorlinks=true,
   allcolors=black,
   pdfstartview=Fit,
   breaklinks=true}
%enables correct jumping to figures when referencing
\usepackage[all]{hypcap}

\newcommand{\commentontext}[2]{\colorbox{yellow!60}{#1}\pdfcomment[color={0.234 0.867 0.211},hoffset=-6pt,voffset=10pt,opacity=0.5]{#2}}
\newcommand{\commentatside}[1]{\pdfcomment[color={0.045 0.278 0.643},icon=Note]{#1}}

%compatibality with packages todo, easy-todo, todonotes
\newcommand{\todo}[1]{\commentatside{#1}}
%compatiblity with package fixmetodonotes
\newcommand{\TODO}[1]{\commentatside{#1}}

%enable \cref{...} and \Cref{...} instead of \ref: Type of reference included in the link
\usepackage[capitalise,nameinlink]{cleveref}
%Nice formats for \cref
\crefname{section}{Sect.}{Sect.}
\Crefname{section}{Section}{Sections}

\usepackage{xspace}
%\newcommand{\eg}{e.\,g.\xspace}
%\newcommand{\ie}{i.\,e.\xspace}
\newcommand{\eg}{e.\,g.,\ }
\newcommand{\ie}{i.\,e.,\ }

%introduce \powerset - hint by http://matheplanet.com/matheplanet/nuke/html/viewtopic.php?topic=136492&post_id=997377
\DeclareFontFamily{U}{MnSymbolC}{}
\DeclareSymbolFont{MnSyC}{U}{MnSymbolC}{m}{n}
\DeclareFontShape{U}{MnSymbolC}{m}{n}{
    <-6>  MnSymbolC5
   <6-7>  MnSymbolC6
   <7-8>  MnSymbolC7
   <8-9>  MnSymbolC8
   <9-10> MnSymbolC9
  <10-12> MnSymbolC10
  <12->   MnSymbolC12%
}{}
\DeclareMathSymbol{\powerset}{\mathord}{MnSyC}{180}

% correct bad hyphenation here
\hyphenation{op-tical net-works semi-conduc-tor}

%% END COPYING HERE

\usepackage{amsmath}
\usepackage{amssymb}
\usepackage{siunitx}


\begin{document}
\title{Database Reconstruction Attack on Public Data}
\titlerunning{Database Reconstruction}
\ifanonymized
\author{Anonymized Author(s)}
\institute{Institute for Anonymous Papers}
\else
\author{Christian Martindale \and Simson Garfinkel}
\institute{Center for Disclosure Avoidance, U.S. Census Bureau}
\fi

\maketitle
\begin{abstract}
Statistical agencies are mandated to publish summary statistics and
micro data while not providing data users with the ability to derive
specific attributes for particular individuals or
establishments. Today these privacy guarantees are assured through the
practical of \emph{Statistical Disclosure Limitation} (SDL)
techniques, such as suppressing cells and generalizing categories in
summary tables. Although these techniques are not be sufficient to
prevent a database reconstruction attack of the sort anticipated by
Dinur and Nissim\cite{noise}, statistical agencies have been slow to
adopt disclosure limitation techniques based on formal privacy
techniques such as differential privacy---presumably because there is
little knowledge of formal privacy techniques within the broader
official statistics community.  This paper discusses how a database
reconstruction attack functions and demonstrate their effectiveness
and efficiency on a toy example. We then show how SDL techniques based
on additive noise can be effective at defending against database
reconstruction.
\end{abstract}

\begin{keywords}
database reconstruction attack, SAT solver, privacy, disclosure avoidance
\end{keywords}


\section{Introduction}
Working Paper \#22 of the U.S. Federal Committee on Statistical
Methodology\cite{workingpaper22} outlines the currently accepted best
practice for statistical agencies to follow when they prepare and
release both statistical data and ``de-identified''
microdata. Broadly, statistical agencies are charged with releasing
high-quality data to further public policy goals, but they are
prohibited from releasing data that might result in the identification
of a data from a specific individual or establishment.

Working Paper \#22 outlines a number of approaches that statistical
agencies can use for protecting respondent data. This techniques include:
\begin{enumerate}
  \item \textbf{Cell Suppression}, where the values of cells with small counts or few possible
        generating combinations are removed from the published table
  \item \textbf{Row Swapping}, where the data rows corresponding to individuals
        with similar values in certain key cells are switched
  \item \textbf{Generalization}, where numerical values are grouped into
        buckets corresponding to ranges, instead of giving the exact
        values for each entry in the table
  \item \textbf{Top-and-bottom-coding}, where the statistical groups at the high and low ends
        of the table are given without upper or lower bound (e.g.
        reporting the highest group for age as 80+ instead of
        80-90 and 90-100)
\end{enumerate}

While it makes intuitive sense that these techniques hamper the
ability of a potential data intruder\cite{data-intruder} to recover respondent data from the
statistical release, such hunches do not constitute rigorous
mathematical proofs. Absent a formal definition of privacy and
mathematical proofs showing that a specific disclosure limitation
technique realize that definition, there is no way to know if these
techniques actually protect privacy, or if that is merely wishful thinking.

Indeed, despite roughly 15 years' passage since Dinur and Nissum's
groundbreaking work\cite{noise}, and 12 years' since the formalization
of differential privacy\cite{Dwork:2006:CNS:2180286.2180305}, today most of
statistical agencies still rely on the disclosure
limitation techniques described in Working Paper \#22. The purpose of
this paper, then, is to show how a database reconstruction attack
might be implemented against a data release from an official
statistics agency by showing a toy example, and then to show how a
formally private  technique can protect against such an attack.

\section{Related Work}

Dinur and Nissim\cite{noise} showed that the underlying
confidential data of a statistical database can be reconstructed with
a surprisingly small number of queries. In part, this is because the
architecture of the released information forms a set of constraints
for which there is ultimately only one exact solution. They further
demonstrated that restricting the number or type of queries is often
insufficient to prevent access to indirectly identifying information,
since the system's refusal to answer a 'dangerous' query itself gives
the user information. The authors found that if a database is modeled
as a string of $n$ bits, then at least $\sqrt{n}$ bits must be
modified by adding noise to protect individuals from being identified. 

In this paper we shall use
the term \emph{database reconstruction attack} (DRA) to describe the process of
taking a published set of statistical tables and deriving the
underlying sensitive data. 

Kasiviswanathan, Rudelson, and Smith\cite{Kasiviswanathan:2013:PLR:2627817.2627919} introduced
the concept of the linear reconstruction attack which is the root
behind the generic DRA. The key concept is that,
given nonsensitive data such as zip code or gender, an attacker
can construct a matrix of linear equalities that can often be solved
in polynomial time. The paper also analyzes a common reconstruction
technique known as least squares decoding, where the attacker sets up
a goal function to minimize the square of the distance between two
databases in order to reconstruct the original database. 

Brown and Heathers\cite{doi:10.1177/1948550616673876} developed the
granularity-related inconsistency of means (GRIM) test in response to
observed inconsistencies in published data from psychological
journals. This test is centered around the premise that, for
statistics drawn from integer data, only certain means are
possible. The GRIM test determines whether reported means could
possibly have come from data sets with a certain size, granularity,
and group number. In surveying 71 published articles, the authors
found 36 papers with one inconsistency and 16 with two or more
inconsistencies. Although this test was intended to detect possible
errors or mean falsification in published articles, the concept of
drawing inferential conclusions about a data set based only off of
published statistics is a key concept behind the DRA.

\section{The Database Reconstruction Attack: An Example}

To present a DRA, we consider a hypothetical
census of a fictional block conducted by a fictional statistical
agency. 

\subsection{The released statistical data}

For every household, the agency collects each member's age,
sex, race, and their generation in the household.  Because the
confidential data will not be released in order to protect the
respondent's privacy, the statistical agency 
publishes every conceivable statistic that might be of use to a
potential demographers, sociologists and economists. Based on user
feedback from previous census reports, the agency has come up with the
a set of statistics that it calls ``Table~\ref{publishedstatsbig},''
and which we present here.

Note that in order to protect respondent privacy, the agency has
suppressed all statistics that would only result from a single
individual. The agency does this even though the statistics are
presumably ``de-identified''---that is, no identifying information is
published with the summary statistics, so the intuition is that there
is no way to match these statistics back up to specific individuals.

\begin{table}
\begin{center}
\begin{tabular}{l|l|c|c}
Item & Group & Number & Average Age \\
\hline
1A & Individuals & 10 & 40 \\
1B & Males & 5 & 34 \\
1C & Females & 5 & 46 \\
1D & Whites & 5 & 50 \\
1E & Blacks & 5 & 30 \\
\hline
2A & Children (0-12) & 3 & 10 \\
2D & White children & 1 & \multicolumn{1}{c}{\rule{6mm}{3mm}} \\
2E & Black children & 2 & 10 \\
\hline
3A & Parents & 4 & 32.5 \\
3B & Male parents & 2 & 30 \\
3C & Female parents & 2 & 35 \\
35 & Parents over 40 & 0 & -- \\
\hline
4A & Grandparents & 3 & 80 \\
4B & Male grandparents & 1 & \multicolumn{1}{c}{\rule{6mm}{3mm}} \\
4C & Female grandparents & 2 & 75 \\
4D & White grandparents & 2 & 80 \\
4E & Black grandparents & 1 & \multicolumn{1}{c}{\rule{6mm}{3mm}} \\
\hline
5A & Households & 2 & 40 \\
5B & Tri-generational households & 0 & -- \\
5C & Single-parent households & 0 & -- \\
5D & Childless households & 1 & \multicolumn{1}{c}{\rule{6mm}{3mm}} \\
5E & Interracial married couples & 2 & 32.5 \\
5F & Same-sex married couples & 0 & -- \\
5G & Households $\geq 40\% $ female & 2 & 40 \\
5H & Households $\geq 40\% $ black & 2 & 40 \\
\hline
\end{tabular}
\end{center}
\caption{Fictional statistical data published by the fictional
  statistics agency. Item numbers are for identification purpose
  only. Note that statistics 2D, 4B, 4E and 5D have been suppressed as part
  of the statistical disclosure limitation process.}\label{publishedstatsbig}
\end{table}

\subsection{Setting up the attack}

The goal of the DRA is to reconstruct the number of households and,
for each household, to learn the age, sex, race, and generation of
each member. The attack views these attributes as a set of five
unknown variables. Since there are 10 individuals in
the fictitious census, there are 50 unknowns. They are shown in
Table~\ref{50unknowns}. 

\begin{table}
\begin{minipage}[t]{3in}
\begin{tabular}{cccccc}
ID & Household & Age & Sex & Race & Generation \\
\hline
\hline
1 & H1 & A1 & S1 & R1 & G1  \\
\hline
2 & H2 & A2 & S2 & R2 & G2  \\
\hline
3 & H3 & A3 & S3 & R3 & G3  \\
\hline
4 & H4 & A4 & S4 & R4 & G4  \\
\hline
5 & H5 & A5 & S5 & R5 & G5  \\
\hline
6 & H6 & A6 & S6 & R6 & G6  \\
\hline
7 & H7 & A7 & S7 & R7 & G7  \\
\hline
8 & H8 & A8 & S8 & R8 & G8  \\
\hline
9 & H9 & A9 & S9 & R9 & G9  \\
\hline
10 & H10 & A10 & S10 & R10 & G10  \\
\hline
\end{tabular}
\end{minipage}
\begin{minipage}[t]{1in}
\begin{tabular}{c|c}
Key & Value \\
\hline
Male & 0 \\
Female & 1 \\
\hline
White & 0 \\
Black & 1 \\
\hline
Child & 0 \\
Parent & 1 \\
Grandparent & 2 \\
\hline
\end{tabular}
\end{minipage}
\caption{Unknowns for the 10 fictitious individuals whose statistical
  data are presented in Table~\ref{publishedstatsbig}. Unknowns
  H\textit{n} are the household, A\textit{n} the age, S\textit{n} the
  sex, R\textit{n} the race and G\textit{n} the generation. Sex, Race
  and Generation categorical values; they are converted into numerical
  values using the key at the right.}\label{50unknowns}
\end{table}

\subsection{Setting up the constraints}

The reconstruction attack works by examining data published by the
statistical agency and identifying how those data can be translated
into constraints on the private data. The constraints are expressed as
mathematical formulae representing rules that the ground truth must
satisfy. In \S\ref{c1}, \S\ref{c2} and \S\ref{c3} we identify how
specific released statistics can be mapped to constraints.
\subsubsection{Constraint 1}\label{c1}

The first constraint relies on statistic 1A:\\

\begin{center}
\begin{tabular}{l|l|c|c}
Item & Group & Number & Average Age \\
\hline
1A & Individuals & 10 & 40 \\
\multicolumn{4}{c}{$\vdots$}
\end{tabular}
\end{center}

This constraint can be written as a linear constraint equation using
the symbols for age, $A1...A10$ given in Table~\ref{50unknowns}:

\[\frac{A1 + A2 + A3 + A4 +...+ A10}{10} = 40\]

This equation has 10 unknowns, each of which has a range from 0..120,
and thus there are many possible solutions.

However, the attacker can write more constraint equations to further constrain the values of the unknowns.

\subsubsection{Constraint 2}\label{c2}

To encode the second constraint we must introduce the == operator,
which here is a test for equivalence. This operator returns 1 if the
left and right hand sides are equal, and 0 otherwise. A truth table
for this operator is shown in Table~\ref{truthtable}. Additionally,
the = operator is used to show that the right-hand side and left-hand
side of this operator must be equal. 

\begin{table}[t]
\begin{center}
\begin{tabular}{cc|c}
A & B & A==B  \\
\hline
True & True & 1   \\
True & False & 0  \\
False & True & 0  \\
False & False & 1  \\
\hline
\end{tabular}
\caption{Truth table for the == operator}\label{truthtable}
\end{center}
\end{table}

The constraint that we will encode is:

\begin{center}
\begin{tabular}{l|l|c|c}
Item & Group & Number & Average Age \\
\hline
\multicolumn{4}{c}{$\vdots$}\\
2A & Children (0-12) & 3 & 10 \\
\multicolumn{4}{c}{$\vdots$}\\
\end{tabular}
\end{center}

This constraint becomes the following two equations, the first of which asserts that
there must be three children, the second of which asserts that their
combined age must be 30, implied by the fact that their average age
is given as 10:

\begin{align*}
& (G1==0) + (G2==0) + (G3==0)+ (G4==0) + (G5==0) + \\
& (G6==0) + (G7==0) + (G8==0)+ (G9==0) + (G10==0) = 3 \\
\\
&  A1 \times (G1==0) + A2 \times (G2==0) + A3 \times (G3==0) + A4 \times (G4==0) + \\
&  A5 \times (G5==0) + A6 \times (G6==0) + A7 \times (G7==0) + A8 \times (G8==0) + \\
&  A9 \times (G9==0) + A10 \times (G10==0) = 30 
\end{align*}




\begin{table}
\begin{tabular}{rllp{1in}||rllp{1in}}
\multicolumn{4}{c||}{Household \#1}    & \multicolumn{4}{c}{Household \#2} \\
Age & Sex & Race & Gen                 & Age & Sex & Race & Gen \\
\hline
90 & Male & White & Grandparent        & 40 & Female & White & Parent\\
80 & Female & Black & Grandparent      & 20 & Male & Black & Parent\\
70 & Female & White & Grandparent      & 10 & Female & White & Child\\
40 & Male & White & Parent             & 10 & Male & Black & Child\\
30 & Female & Black & Parent           & 10 & Male & Black & Child\\
\hline
\end{tabular}
\caption{Responses from a two fictional households
for a survey that collects Age, Sex, Race and Generation of each resident. This is
the unpublished, confidential data collected by a fictional statistical
agency.}\label{responses}
\end{table}


\begin{table}
\begin{tabular}{c|c|c|c|c|c}
ID & Household & Age & Sex & Race & Generation \\
\hline
1 & 1 & 80 & 1 & 1 & 2  \\
2 & 1 & 40 & 0 & 0 & 1  \\
3 & 1 & 70 & 1 & 0 & 2  \\
4 & 1 & 30 & 1 & 1 & 1  \\
5 & 1 & 90 & 0 & 0 & 2  \\
6 & 2 & 10 & 0 & 1 & 0  \\
7 & 2 & 10 & 0 & 1 & 0  \\
8 & 2 & 10 & 1 & 0 & 0  \\
9 & 2 & 40 & 1 & 0 & 1 \\
10 & 2 & 20 & 0 & 1 & 1 \\
\hline
\end{tabular}
\caption{Survey results}
\label{resultsbig}
\end{table}

\subsubsection{Filling in the Remaining Constraints}\label{c3}
Once the attacker has converted all the published statistics
into equations like the ones above, the equations together form a
system of many equations in 50 unknowns. 
There is a \textit{solution universe} of all possible solutions to
this set of simultaneous equations. If there is a single possible
solution, then the published statistics completely reveal the
underlying confidential data---the \emph{ground truth}. However, if
there is more than one possible solution, then one of those solutions
is correct, and the others are not. 

\section{Performing the Attack}
A straight forward but possibly inefficient approach to solving this
system of simultaneous equations is to perform a brute force
search---that is, to try all possible values of all possible variables
and record the ones that work.

Ordinarily a brute force search on even this toy problem would be
unreasonable: there are 10 age variables that can range from 0...120,
30 binary variables (household, sex and race), and ten trinary
variable. All together, there are $120^{10} \times 2^{30} \times
3^{10} \approx 4 \times 10^{34}$ possible combinations. Even if we
could try a billion a second, trying all possible combinations would
take a billion billion years.

Those familiar with complexity theory will have recognized by now that
the system of variables can be expressed as a satisfiability (SAT)
problem purely Boolean variables. This is done by encoding each of the
integer variables and trinary variables as themselves being derived
from a set of Boolean variables with still more constraints. Encoding
the DRA has a SAT problem allows it to be solved using a SAT Solver,
programs that have been developed over the past two
decades that use sophisticated heuristics that can rapidly solve many
SAT problems. In general, SAT solvers work by identifying a ``kernel''
of constraints that must always be true (or, conversely, can never be
true) and using these kernels to systematically eliminate large parts
of the search space. In recent years SAT solvers have been
successfully applied to many applications.

\textbf{SIMSON'S REWRITE IS UP TO HERE}

From this data, the attacker generates many constraint equations as demonstrated earlier,
and then inputs these constraints into a SAT solver. In this example, we use a free open-source SAT solver
called PicoSAT written in C. PicoSAT takes input in the standard DIMACS file format. Because the DIMACS format can be difficult to understand, we use another open source program, Sugar\cite{sugar}, to translate our system of constraints into the DIMACS format. In Sugar, constraints are given to the SAT solver through \textit{s-expressions}, which are mathematical expressions written in a special nested prefix form. For example, the constraint "Average total age (in a population of size 5): 40" is encoded as the following s-expression:
\begin{verbatim}
(= (/ (+ A1 A2 A3 A4 A5) 5) 40)
\end{verbatim}

S-expressions are evaluated from the innermost parentheses to the outermost, with the operator at the left of each parenthetic expression being applied to the rest of the arguments in that expression. The above expression is evaluated using the following steps:
\begin{enumerate}
    \item Take the sum of variables A1...A5.
    \item Divide the result of Step 1 by 5.
    \item The result of Step 2 must equal 40.
\end{enumerate}

Output is given in Table~\ref{sugarbig} and is tabulated for readability. Although the ID numbers are permutated, the ten people in the SAT solver output are identical to the ten people from the ground truth in Table~\ref{resultsbig}, so the attack has succeeded in reconstructing the entire database and narrowing down the solution universe to size 1. Finding this solution required 35 seconds on a single 2012 MacBook Pro with an Intel i7 2.9GHz processor.

At past international SAT solver competitions, SAT solvers such as PicoSAT have solved problems with tens of millions of variables in less than 20 minutes \cite{satcomp}, demonstrating that dataset size and constraint number is not an insurmountable obstacle to this type of DRA.

\begin{table}
\begin{tabular}{c|c|c|c|c|c}
ID & Household & Age & Sex & Race & Generation \\
\hline
1 & 2 & 10 & 1 & 0 & 0  \\
2 & 2 & 20 & 0 & 1 & 1  \\
3 & 1 & 30 & 1 & 1 & 1  \\
4 & 2 & 10 & 0 & 1 & 0  \\
5 & 2 & 40 & 1 & 0 & 1  \\
6 & 2 & 10 & 0 & 1 & 0  \\
7 & 1 & 40 & 0 & 0 & 1  \\
8 & 1 & 80 & 1 & 1 & 2  \\
9 & 1 & 90 & 0 & 0 & 2 \\
10 & 1 & 70 & 1 & 0 & 2 \\
\hline
\end{tabular}
\caption{Sugar output when run on the encoded statistics in Appendix 3}\label{sugarbig}
\end{table}


\section{Defending Against a DRA}
As demonstrated above, new techniques must be developed in
order to protect databases against reconstruction attacks, as traditional techniques such as cell suppression are insufficient defense. One of the simplest and most effective techniques in defending against the DRA is noise infusion, where the publishing agency adds random values to data before publication in order to increase the size of the solution universe. For example, if the statistical agency takes a true value of age = 10, then randomly adds either -2, -1, 0, 1, or 2, and then calculates and publishes statistics consistent with this new age, there are now five elements in the solution universe for this value, where without noise infusion there would be only one. If two responses are processed with noise of this sort, there are now $5^2 = 25$ possible solutions. If 10 people undergo noise infusion of this type, there are $5^{10}$ --- nearly 10 million --- possible solutions. Notice that the statistical agency can reveal that noise infusion has taken place, and can even reveal how much noise has been added, without compromising the effectiveness of said noise infusion. Noise infusion allows an agency to publish more statistics without fearing that it has given attackers too many constraints to work with.

In the following example of noise infusion, we use noise generated
from a distribution called the Laplace distribution
\cite{Dwork:2006:CNS:2180286.2180305}. We choose this distribution
because, when correctly applied, this type of noise addition gives a
strong mathematical guarantee of individual privacy. To demonstrate
how effective adding Laplace noise is against the DRA, we reconsider
the example from the previous section. This time, the statistical
agency applies the Laplace transformation with mean 0 and exponential
decay 3 to the survey responses to the ages before publishing any
statistical products. The new published data product, analogous to
Table~\ref{publishedstatsbig}, is shown in
Table~\ref{publishedstatsnoise}.

\begin{table}[t]
\begin{tabular}{c|c|c}
Group & Number & Average Age \\
\hline
Individuals & 10 & 38.57 \\
Males & 5 & 33.21 \\
Females & 5 & 43.92 \\
Whites & 5 & 47.82 \\
Blacks & 5 & 29.31 \\
\hline
Children (0-12) & 3 & 8.50 \\
White children & 1 & \multicolumn{1}{c}{\rule{6mm}{3mm}} \\
Black children & 2 & 8.75 \\
\hline
Parents & 4 & 31.41 \\
Male parents & 2 & 29.77 \\
Female parents & 2 & 33.06 \\
Parents over 40 & 0 & -- \\
\hline
Grandparents & 3 & 78.18 \\
White grandparents & 2 & 77.64 \\
Black grandparents & 1 & \multicolumn{1}{c}{\rule{6mm}{3mm}} \\
Male grandparents & 1 & \multicolumn{1}{c}{\rule{6mm}{3mm}} \\
Female grandparents & 2 & 72.76 \\
\hline
Households & 2 & 38.57 \\
Tri-generational households & 0 & -- \\
Single-parent households & 0 & -- \\
Childless households & 1 & \multicolumn{1}{c}{\rule{6mm}{3mm}} \\
Interracial married couples & 2 & 31.41 \\
Same-sex married couples & 0 & -- \\
Households $\geq 40\% $ female & 2 & 38.57 \\
Households $\geq 40\% $ black & 2 & 38.57 \\

\hline
\end{tabular}
\caption{Data publication with added Laplace noise}\label{publishedstatsnoise}
\end{table}
It is important to note that the infusion of the noise did not significantly impact the reported statistics, thus preserving their utility to nonintrusive users. However, when the attacker attempts to perform the DRA using the same process as before, Sugar returns "Unsatisfiable". The noise infusion has defeated the DRA, and the statistical agency has upheld its legal obligation to protect respondent privacy.

\section{Problem Analysis}

With the dramatic improvement in the efficiency of SAT solvers in the last decade, the database reconstruction attack is no longer a solely theoretical danger. The vast amount of data products the Census publishes each year gives a determined and informed attacker more than enough constraints to reconstruct some or all of a target database and breach the privacy of millions of people. Although the Census Bureau's current disclosure avoidance technique suite is often sufficient to defend against a cursory attempt at an identification attack, we have demonstrated through attack simulations that cell suppression and grouping are not completely against the power of more sophisticated forms of attack.

The most effective technique for defending against a DRA is the infusion of noise to results before publication. If the noise is generated by correctly applying the Laplace mechanism, it is possible to guarantee individuals' choice to respond to a survey cannot harm them while simultaneously preserving the accuracy of the data for benign use. In infusion to continuing to use the traditional disclosure avoidance techniques, data agencies should make probabilistic noise infusion a required part of the pre-publication disclosure review process in order to protect respondent privacy.

\section{References}

\bibliographystyle{splncs03}
\bibliography{white_paper}


\section{Appendices}

\subsection{SAT and SAT Solvers}

The Boolean satisfiability problem, known as SAT, was the first problem to be proven NP-complete \cite{cooklevin}. This problem asks, for a given Boolean formula, whether replacing each variable with either True or False can make the formula evaluate to True.   A consequence of SAT being NP-complete is that any problem can be reduced in polynomial time (i.e., quickly) to an instance of the SAT problem. Once a problem has been reduced to an instance of the SAT problem, this SAT problem can be fed into a SAT solver to find possible solutions. Although the SAT problem is not solvable by algorithms in polynomial time, researchers have found many heuristic techniques for expediting this process. SAT solvers combine a variety of these techniques into one complex process, resulting in polynomial time solutions for the SAT problem in many cases.

Modern SAT solvers use a heuristic technique called Conflict-Driven Clause Learning, commonly referred to as CDCL. CDCL works by the following process \cite{cdcl}:

\begin{enumerate}

\item Assign a value to a variable arbitrarily.
\item Use this assignment to determine values for the other variables in the formula (a process known as unit propagation).
\item If a conflict is found, backtrack to the clause that made the conflict occur and undo variable assignments made after that point.
\item Add the negation of the conflict-causing clause as a new clause to the master formula and resume from step 1.

\end{enumerate}

This process is much faster at solving SAT problems than previous processes used in SAT solvers because adding conflicts as new clauses has the potential to avoid wasteful 'repeated backtracks'. Additionally, CDCL and its predecessor algorithm, DPLL, are both provably complete algorithms and will always return either a solution or "Unsatisfiable" if given enough time and memory.

There are a wide variety of SAT solvers available to the public for minimal or no cost. Although a SAT solver requires the user to translate the problem into Boolean formulae before use,  programs such as Naoyuki Tamura's Sugar facilitate this process by translating user-input mathematical and English constraints into Boolean formulae automatically.



\subsection{Sugar Input}

Sugar input is given in a standard Constraint Satisfaction Problem (CSP) file format. A constraint must be given on a single line of the file, but here we separate most constraints into multiple lines for readability. Constraint equations are separated by comments describing what statistics they encode.

Input for the mock survey from Section 4 is as follows:

\begin{verbatim}
;define variables and their domains
;households (either 1 or 2)
(int H1 1 2)
(int H2 1 2)
(int H3 1 2)
(int H4 1 2)
(int H5 1 2)
(int H6 1 2)
(int H7 1 2)
(int H8 1 2)
(int H9 1 2)
(int H10 1 2)
;ages (between 1 and 90 years old)
(int A1 1 90)
(int A2 1 90)
(int A3 1 90)
(int A4 1 90)
(int A5 1 90)
(int A6 1 90)
(int A7 1 90)
(int A8 1 90)
(int A9 1 90)
(int A10 1 90)
;sexes (male or female)
(int S1 0 1)
(int S2 0 1)
(int S3 0 1)
(int S4 0 1)
(int S5 0 1)
(int S6 0 1)
(int S7 0 1)
(int S8 0 1)
(int S9 0 1)
(int S10 0 1)
;races (white or black)
(int R1 0 1)
(int R2 0 1)
(int R3 0 1)
(int R4 0 1)
(int R5 0 1)
(int R6 0 1)
(int R7 0 1)
(int R8 0 1)
(int R9 0 1)
(int R10 0 1)
;generations (child, parent, or grandparent)
(int G1 0 2)
(int G2 0 2)
(int G3 0 2)
(int G4 0 2)
(int G5 0 2)
(int G6 0 2)
(int G7 0 2)
(int G8 0 2)
(int G9 0 2)
(int G10 0 2)
;
;avg age total
(= (/ (+ A1 A2 A3 A4 A5 A6 A7 A8 A9 A10)
10) 40)
;household sizes
(<= (+ (if (= H1 1) 1 0) (if (= H2 1) 1 0)
(if (= H3 1) 1 0) (if (= H4 1) 1 0)
(if (= H5 1) 1 0) (if (= H6 1) 1 0)
(if (= H7 1) 1 0) (if (= H8 1) 1 0)
(if (= H9 1) 1 0) (if (= H10 1) 1 0) ) 5)
(>= (+ (if (= H1 1) 1 0) (if (= H2 1) 1 0)
(if (= H3 1) 1 0) (if (= H4 1) 1 0)
(if (= H5 1) 1 0) (if (= H6 1) 1 0)
(if (= H7 1) 1 0) (if (= H8 1) 1 0)
(if (= H9 1) 1 0) (if (= H10 1) 1 0) ) 5)
;number of children
(= (+ (if (= G1 0) 1 0) (if (= G2 0) 1 0)
(if (= G3 0) 1 0) (if (= G4 0) 1 0)
(if (= G5 0) 1 0) (if (= G6 0) 1 0)
(if (= G7 0) 1 0) (if (= G8 0) 1 0)
(if (= G9 0) 1 0) (if (= G10 0) 1 0) ) 3)
;number of parents
(= (+ (if (= G1 1) 1 0) (if (= G2 1) 1 0)
(if (= G3 1) 1 0) (if (= G4 1) 1 0)
(if (= G5 1) 1 0) (if (= G6 1) 1 0)
(if (= G7 1) 1 0) (if (= G8 1) 1 0)
(if (= G9 1) 1 0) (if (= G10 1) 1 0) ) 4)
;number of grandparents
(= (+ (if (= G1 2) 1 0) (if (= G2 2) 1 0)
(if (= G3 2) 1 0) (if (= G4 2) 1 0)
(if (= G5 2) 1 0) (if (= G6 2) 1 0)
(if (= G7 2) 1 0) (if (= G8 2) 1 0)
(if (= G9 2) 1 0) (if (= G10 2) 1 0) ) 3)
;number of males
(= (+ (if (= S1 0) 1 0) (if (= S2 0) 1 0)
(if (= S3 0) 1 0) (if (= S4 0) 1 0)
(if (= S5 0) 1 0) (if (= S6 0) 1 0)
(if (= S7 0) 1 0) (if (= S8 0) 1 0)
(if (= S9 0) 1 0) (if (= S10 0) 1 0) ) 5)
;number of females
(= (+ (if (= S1 1) 1 0) (if (= S2 1) 1 0)
(if (= S3 1) 1 0) (if (= S4 1) 1 0)
(if (= S5 1) 1 0) (if (= S6 1) 1 0)
(if (= S7 1) 1 0) (if (= S8 1) 1 0)
(if (= S9 1) 1 0) (if (= S10 1) 1 0) ) 5)
;number of blacks
(= (+ (if (= R1 1) 1 0) (if (= R2 1) 1 0)
(if (= R3 1) 1 0) (if (= R4 1) 1 0)
(if (= R5 1) 1 0) (if (= R6 1) 1 0)
(if (= R7 1) 1 0) (if (= R8 1) 1 0)
(if (= R9 1) 1 0) (if (= R10 1) 1 0) ) 5)
;number of whites
(= (+ (if (= R1 0) 1 0) (if (= R2 0) 1 0)
(if (= R3 0) 1 0) (if (= R4 0) 1 0)
(if (= R5 0) 1 0) (if (= R6 0) 1 0)
(if (= R7 0) 1 0) (if (= R8 0) 1 0)
(if (= R9 0) 1 0) (if (= R10 0) 1 0) ) 5)
;avg age blacks
(= (/ (* (+ (* A1 (if (= R1 1) 1 0))
(* A2 (if (= R2 1) 1 0))
(* A3 (if (= R3 1) 1 0))
(* A4 (if (= R4 1) 1 0))
(* A5 (if (= R5 1) 1 0))
(* A6 (if (= R6 1) 1 0))
(* A7 (if (= R7 1) 1 0))
(* A8 (if (= R8 1) 1 0))
(* A9 (if (= R9 1) 1 0))
(* A10 (if (= R10 1) 1 0))) 10) 5) 300)
;avg age whites
(= (/ (* (+ (* A1 (if (= R1 0) 1 0))
(* A2 (if (= R2 0) 1 0))
(* A3 (if (= R3 0) 1 0))
(* A4 (if (= R4 0) 1 0))
(* A5 (if (= R5 0) 1 0))
(* A6 (if (= R6 0) 1 0))
(* A7 (if (= R7 0) 1 0))
(* A8 (if (= R8 0) 1 0))
(* A9 (if (= R9 0) 1 0))
(* A10 (if (= R10 0) 1 0)))  10) 5) 500)
;avg age children
(= (/ (* (+ (* A1 (if (= G1 0) 1 0))
(* A2 (if (= G2 0) 1 0))
(* A3 (if (= G3 0) 1 0))
(* A4 (if (= G4 0) 1 0))
(* A5 (if (= G5 0) 1 0))
(* A6 (if (= G6 0) 1 0))
(* A7 (if (= G7 0) 1 0))
(* A8 (if (= G8 0) 1 0))
(* A9 (if (= G9 0) 1 0))
(* A10 (if (= G10 0) 1 0)) ) 10) 3) 100)
;avg age parents
(= (/ (* (+ (* A1 (if (= G1 1) 1 0))
(* A2 (if (= G2 1) 1 0))
(* A3 (if (= G3 1) 1 0))
(* A4 (if (= G4 1) 1 0))
(* A5 (if (= G5 1) 1 0))
(* A6 (if (= G6 1) 1 0))
(* A7 (if (= G7 1) 1 0))
(* A8 (if (= G8 1) 1 0))
(* A9 (if (= G9 1) 1 0))
(* A10 (if (= G10 1) 1 0))) 10) 4) 325)
;avg age grandparents
(= (/ (* (+ (* A1 (if (= G1 2) 1 0))
(* A2 (if (= G2 2) 1 0))
(* A3 (if (= G3 2) 1 0))
(* A4 (if (= G4 2) 1 0))
(* A5 (if (= G5 2) 1 0))
(* A6 (if (= G6 2) 1 0))
(* A7 (if (= G7 2) 1 0))
(* A8 (if (= G8 2) 1 0))
(* A9 (if (= G9 2) 1 0))
(* A10 (if (= G10 2) 1 0)) ) 10) 3) 800)
;avg age males
(= (/ (* (+ (* A1 (if (= S1 0) 1 0))
(* A2 (if (= S2 0) 1 0))
(* A3 (if (= S3 0) 1 0))
(* A4 (if (= S4 0) 1 0))
(* A5 (if (= S5 0) 1 0))
(* A6 (if (= S6 0) 1 0))
(* A7 (if (= S7 0) 1 0))
(* A8 (if (= S8 0) 1 0))
(* A9 (if (= S9 0) 1 0))
(* A10 (if (= S10 0) 1 0)) ) 10) 5) 340)
;avg age females
(= (/ (* (+ (* A1 (if (= S1 1) 1 0))
(* A2 (if (= S2 1) 1 0))
(* A3 (if (= S3 1) 1 0))
(* A4 (if (= S4 1) 1 0))
(* A5 (if (= S5 1) 1 0))
(* A6 (if (= S6 1) 1 0))
(* A7 (if (= S7 1) 1 0))
(* A8 (if (= S8 1) 1 0))
(* A9 (if (= S9 1) 1 0))
(* A10 (if (= S10 1) 1 0)) ) 10) 5) 460)
;number of households
(nvalue 2 (H1 H2 H3 H4 H5 H6 H7 H8 H9 H10))
;number of male children
(= (+ (if (and (= G1 0) (= S1 0 )) 1 0)
(if (and (= G2 0) (= S2 0 )) 1 0)
(if (and (= G3 0) (= S3 0 )) 1 0)
(if (and (= G4 0) (= S4 0 )) 1 0)
(if (and (= G5 0) (= S5 0 )) 1 0)
(if (and (= G6 0) (= S6 0 )) 1 0)
(if (and (= G7 0) (= S7 0 )) 1 0)
(if (and (= G8 0) (= S8 0 )) 1 0)
(if (and (= G9 0) (= S9 0 )) 1 0)
(if (and (= G10 0) (= S10 0 )) 1 0) ) 2)
;number of female children
(= (+ (if (and (= G1 0) (= S1 1 )) 1 0)
(if (and (= G2 0) (= S2 1 )) 1 0)
(if (and (= G3 0) (= S3 1 )) 1 0)
(if (and (= G4 0) (= S4 1 )) 1 0)
(if (and (= G5 0) (= S5 1 )) 1 0)
(if (and (= G6 0) (= S6 1 )) 1 0)
(if (and (= G7 0) (= S7 1 )) 1 0)
(if (and (= G8 0) (= S8 1 )) 1 0)
(if (and (= G9 0) (= S9 1 )) 1 0)
(if (and (= G10 0) (= S10 1 )) 1 0) ) 1)
;number of children 0-12: 3
(= (+ (if (and (>= A1 1) (<= A1 12)) 1 0)
(if (and (>= A2 1) (<= A2 12)) 1 0)
(if (and (>= A3 1) (<= A3 12)) 1 0)
(if (and (>= A4 1) (<= A4 12)) 1 0)
(if (and (>= A5 1) (<= A5 12)) 1 0)
(if (and (>= A6 1) (<= A6 12)) 1 0)
(if (and (>= A7 1) (<= A7 12)) 1 0)
(if (and (>= A8 1) (<= A8 12)) 1 0)
(if (and (>= A9 1) (<= A9 12)) 1 0)
(if (and (>= A10 1) (<= A10 12)) 1 0) ) 3)
;number of children 13-17: 0
(= (+ (if (and (>= A1 13) (<= A1 17)) 1 0)
(if (and (>= A2 13) (<= A2 17)) 1 0)
(if (and (>= A3 13) (<= A3 17)) 1 0)
(if (and (>= A4 13) (<= A4 17)) 1 0)
(if (and (>= A5 13) (<= A5 17)) 1 0)
(if (and (>= A6 13) (<= A6 17)) 1 0)
(if (and (>= A7 13) (<= A7 17)) 1 0)
(if (and (>= A8 13) (<= A8 17)) 1 0)
(if (and (>= A9 13) (<= A9 17)) 1 0)
(if (and (>= A10 13) (<= A10 17)) 1 0) ) 0)
;avg age of children 0-12: 10
(= (/ (* (+
(* A1 (if (and (>= A1 1) (<= A1 12)) 1 0))
(* A2 (if (and (>= A2 1) (<= A2 12)) 1 0))
(* A3 (if (and (>= A3 1) (<= A3 12)) 1 0))
(* A4 (if (and (>= A4 1) (<= A4 12)) 1 0))
(* A5 (if (and (>= A5 1) (<= A5 12)) 1 0))
(* A6 (if (and (>= A6 1) (<= A6 12)) 1 0))
(* A7 (if (and (>= A7 1) (<= A7 12)) 1 0))
(* A8 (if (and (>= A8 1) (<= A8 12)) 1 0))
(* A9 (if (and (>= A9 1) (<= A9 12)) 1 0))
(* A10 (if (and (>= A10 1) (<= A10 12)) 1 0)))
10) 3) 100)
;number of hh with 3 or more generations: 1
(xor (nvalue 3 (G1 G2 G3 G4 G5))
(nvalue 3(G6 G7 G8 G9 G10)))
;number of male parents
(= (+ (if (and (= G1 1) (= S1 0 )) 1 0)
(if (and (= G2 1) (= S2 0 )) 1 0)
(if (and (= G3 1) (= S3 0 )) 1 0)
(if (and (= G4 1) (= S4 0 )) 1 0)
(if (and (= G5 1) (= S5 0 )) 1 0)
(if (and (= G6 1) (= S6 0 )) 1 0)
(if (and (= G7 1) (= S7 0 )) 1 0)
(if (and (= G8 1) (= S8 0 )) 1 0)
(if (and (= G9 1) (= S9 0 )) 1 0)
(if (and (= G10 1) (= S10 0 )) 1 0) ) 2)
;number of female parents
(= (+ (if (and (= G1 1) (= S1 1 )) 1 0)
(if (and (= G2 1) (= S2 1 )) 1 0)
(if (and (= G3 1) (= S3 1 )) 1 0)
(if (and (= G4 1) (= S4 1 )) 1 0)
(if (and (= G5 1) (= S5 1 )) 1 0)
(if (and (= G6 1) (= S6 1 )) 1 0)
(if (and (= G7 1) (= S7 1 )) 1 0)
(if (and (= G8 1) (= S8 1 )) 1 0)
(if (and (= G9 1) (= S9 1 )) 1 0)
(if (and (= G10 1) (= S10 1 )) 1 0) ) 2)
;number of single-parent households: 0
(< (+ (if (and (= H1 1) (= G1 1)) 1 0)
(if (and (= H2 1) (= G2 1)) 1 0)
(if (and (= H3 1) (= G3 1)) 1 0)
(if (and (= H4 1) (= G4 1)) 1 0)
(if (and (= H5 1) (= G5 1)) 1 0)
(if (and (= H6 1) (= G6 1)) 1 0)
(if (and (= H7 1) (= G7 1)) 1 0)
(if (and (= H8 1) (= G8 1)) 1 0)
(if (and (= H9 1) (= G9 1)) 1 0)
(if (and (= H10 1) (= G10 1)) 1 0) ) 3)
(< (+ (if (and (= H1 2) (= G1 1)) 1 0)
(if (and (= H2 2) (= G2 1)) 1 0)
(if (and (= H3 2) (= G3 1)) 1 0)
(if (and (= H4 2) (= G4 1)) 1 0)
(if (and (= H5 2) (= G5 1)) 1 0)
(if (and (= H6 2) (= G6 1)) 1 0)
(if (and (= H7 2) (= G7 1)) 1 0)
(if (and (= H8 2) (= G8 1)) 1 0)
(if (and (= H9 2) (= G9 1)) 1 0)
(if (and (= H10 2) (= G10 1)) 1 0) ) 3)
;total number of black children: 2
(= (+ (if (and (= G1 0) (= R1 1)) 1 0)
(if (and (= G2 0) (= R2 1)) 1 0)
(if (and (= G3 0) (= R3 1)) 1 0)
(if (and (= G4 0) (= R4 1)) 1 0)
(if (and (= G5 0) (= R5 1)) 1 0)
(if (and (= G6 0) (= R6 1)) 1 0)
(if (and (= G7 0) (= R7 1)) 1 0)
(if (and (= G8 0) (= R8 1)) 1 0)
(if (and (= G9 0) (= R9 1)) 1 0)
(if (and (= G10 0) (= R10 1)) 1 0) ) 2)
;total number of black females: 2
(= (+ (if (and (= S1 1) (= R1 1)) 1 0)
(if (and (= S2 1) (= R2 1)) 1 0)
(if (and (= S3 1) (= R3 1)) 1 0)
(if (and (= S4 1) (= R4 1)) 1 0)
(if (and (= S5 1) (= R5 1)) 1 0)
(if (and (= S6 1) (= R6 1)) 1 0)
(if (and (= S7 1) (= R7 1)) 1 0)
(if (and (= S8 1) (= R8 1)) 1 0)
(if (and (= S9 1) (= R9 1)) 1 0)
(if (and (= S10 1) (= R10 1)) 1 0) ) 2)
;all households are at least 40% minority
(>= (+ (if (and (= H1 1) (= R1 1)) 1 0)
(if (and (= H2 1) (= R2 1)) 1 0)
(if (and (= H3 1) (= R3 1)) 1 0)
(if (and (= H4 1) (= R4 1)) 1 0)
(if (and (= H5 1) (= R5 1)) 1 0)
(if (and (= H6 1) (= R6 1)) 1 0)
(if (and (= H7 1) (= R7 1)) 1 0)
(if (and (= H8 1) (= R8 1)) 1 0)
(if (and (= H9 1) (= R9 1)) 1 0)
(if (and (= H10 1) (= R10 1)) 1 0) ) 2)
(>= (+ (if (and (= H1 2) (= R1 1)) 1 0)
(if (and (= H2 2) (= R2 1)) 1 0)
(if (and (= H3 2) (= R3 1)) 1 0)
(if (and (= H4 2) (= R4 1)) 1 0)
(if (and (= H5 2) (= R5 1)) 1 0)
(if (and (= H6 2) (= R6 1)) 1 0)
(if (and (= H7 2) (= R7 1)) 1 0)
(if (and (= H8 2) (= R8 1)) 1 0)
(if (and (= H9 2) (= R9 1)) 1 0)
(if (and (= H10 2) (= R10 1)) 1 0) ) 2)
;all households are at least 40% female
(>= (+ (if (and (= H1 1) (= S1 1)) 1 0)
(if (and (= H2 1) (= S2 1)) 1 0)
(if (and (= H3 1) (= S3 1)) 1 0)
(if (and (= H4 1) (= S4 1)) 1 0)
(if (and (= H5 1) (= S5 1)) 1 0)
(if (and (= H6 1) (= S6 1)) 1 0)
(if (and (= H7 1) (= S7 1)) 1 0)
(if (and (= H8 1) (= S8 1)) 1 0)
(if (and (= H9 1) (= S9 1)) 1 0)
(if (and (= H10 1) (= S10 1)) 1 0) ) 2)
(>= (+ (if (and (= H1 2) (= S1 1)) 1 0)
(if (and (= H2 2) (= S2 1)) 1 0)
(if (and (= H3 2) (= S3 1)) 1 0)
(if (and (= H4 2) (= S4 1)) 1 0)
(if (and (= H5 2) (= S5 1)) 1 0)
(if (and (= H6 2) (= S6 1)) 1 0)
(if (and (= H7 2) (= S7 1)) 1 0)
(if (and (= H8 2) (= S8 1)) 1 0)
(if (and (= H9 2) (= S9 1)) 1 0)
(if (and (= H10 2) (= S10 1)) 1 0) ) 2)
;number of households with
same-sex married couples: 0
(= 1 (+ (if (and (= H1 1)
    (and (= G1 1) (= S1 0))) 1 0)
(if (and (= H2 1)
    (and (= G2 1) (= S2 0))) 1 0)
(if (and (= H3 1)
    (and (= G3 1) (= S3 0))) 1 0)
(if (and (= H4 1)
    (and (= G4 1) (= S4 0))) 1 0)
(if (and (= H5 1)
    (and (= G5 1) (= S5 0))) 1 0)
(if (and (= H6 1)
    (and (= G6 1) (= S6 0))) 1 0)
(if (and (= H7 1)
    (and (= G7 1) (= S7 0))) 1 0)
(if (and (= H8 1)
    (and (= G8 1) (= S8 0))) 1 0)
(if (and (= H9 1)
    (and (= G9 1) (= S9 0))) 1 0)
(if (and (= H10 1)
    (and (= G10 1) (= S10 0))) 1 0)))
(= 1 (+ (if (and (= H1 2)
   (and (= G1 1) (= S1 0))) 1 0)
(if (and (= H2 2)
   (and (= G2 1) (= S2 0))) 1 0)
(if (and (= H3 2)
   (and (= G3 1) (= S3 0))) 1 0)
(if (and (= H4 2)
   (and (= G4 1) (= S4 0))) 1 0)
(if (and (= H5 2)
   (and (= G5 1) (= S5 0))) 1 0)
(if (and (= H6 2)
   (and (= G6 1) (= S6 0))) 1 0)
(if (and (= H7 2)
   (and (= G7 1) (= S7 0))) 1 0)
(if (and (= H8 2)
   (and (= G8 1) (= S8 0))) 1 0)
(if (and (= H9 2)
   (and (= G9 1) (= S9 0))) 1 0)
(if (and (= H10 2)
   (and (= G10 1) (= S10 0))) 1 0)))
;avg age of male parents: 30
(= (/ (* (+ (* (if (and (= G1 1)
(= S1 0)) 1 0) A1)
(* (if (and (= G2 1)
(= S2 0)) 1 0) A2)
(* (if (and (= G3 1)
(= S3 0)) 1 0) A3)
(* (if (and (= G4 1)
(= S4 0)) 1 0) A4)
(* (if (and (= G5 1)
(= S5 0)) 1 0) A5)
(* (if (and (= G6 1)
(= S6 0)) 1 0) A6)
(* (if (and (= G7 1)
(= S7 0)) 1 0) A7)
(* (if (and (= G8 1)
(= S8 0)) 1 0) A8)
(* (if (and (= G9 1)
(= S9 0)) 1 0) A9)
(* (if (and (= G10 1)
(= S10 0)) 1 0) A10)) 10) 2) 300)
;avg age of female parents: 35
(= (/ (* (+ (* (if (and (= G1 1)
   (= S1 1)) 1 0) A1)
(* (if (and (= G2 1)
   (= S2 1)) 1 0) A2)
(* (if (and (= G3 1)
   (= S3 1)) 1 0) A3)
(* (if (and (= G4 1)
   (= S4 1)) 1 0) A4)
(* (if (and (= G5 1)
   (= S5 1)) 1 0) A5)
(* (if (and (= G6 1)
   (= S6 1)) 1 0) A6)
(* (if (and (= G7 1)
   (= S7 1)) 1 0) A7)
(* (if (and (= G8 1)
   (= S8 1)) 1 0) A8)
(* (if (and (= G9 1)
   (= S9 1)) 1 0) A9)
(* (if (and (= G10 1)
   (= S10 1)) 1 0) A10)) 10) 2) 350)
;avg age of female grandparents: 75
(= (/ (* (+ (* (if (and (= G1 2)
(= S1 1)) 1 0) A1)
(* (if (and (= G2 2) (= S2 1)) 1 0) A2)
(* (if (and (= G3 2) (= S3 1)) 1 0) A3)
(* (if (and (= G4 2) (= S4 1)) 1 0) A4)
(* (if (and (= G5 2) (= S5 1)) 1 0) A5)
(* (if (and (= G6 2) (= S6 1)) 1 0) A6)
(* (if (and (= G7 2) (= S7 1)) 1 0) A7)
(* (if (and (= G8 2) (= S8 1)) 1 0) A8)
(* (if (and (= G9 2) (= S9 1)) 1 0) A9)
(* (if (and (= G10 2) (= S10 1)) 1 0) A10))
10) 2) 750)
;number of parents over 40: 0
(= 0 (+ (if (and(= G1 1) (> A1 40)) 1 0)
(if (and(= G2 1) (> A2 40)) 1 0)
(if (and(= G3 1) (> A3 40)) 1 0)
(if (and(= G4 1) (> A4 40)) 1 0)
(if (and(= G5 1) (> A5 40)) 1 0)
(if (and(= G6 1) (> A6 40)) 1 0)
(if (and(= G7 1) (> A7 40)) 1 0)
(if (and(= G8 1) (> A8 40)) 1 0)
(if (and(= G9 1) (> A9 40)) 1 0)
(if (and(= G10 1) (> A10 40)) 1 0)))
;number of children over 10: 0
(= 0 (+ (if (and (= G1 0) (> A1 10)) 1 0)
(if (and (= G2 0) (> A2 10)) 1 0)
(if (and (= G3 0) (> A3 10)) 1 0)
(if (and (= G4 0) (> A4 10)) 1 0)
(if (and (= G5 0) (> A5 10)) 1 0)
(if (and (= G6 0) (> A6 10)) 1 0)
(if (and (= G7 0) (> A7 10)) 1 0)
(if (and (= G8 0) (> A8 10)) 1 0)
(if (and (= G9 0) (> A9 10)) 1 0)
(if (and (= G10 0) (> A10 10)) 1 0)) )
;number of female grandparents: 2
(= (+ (if (and (= G1 2) (= S1 1 )) 1 0)
(if (and (= G2 2) (= S2 1 )) 1 0)
(if (and (= G3 2) (= S3 1 )) 1 0)
(if (and (= G4 2) (= S4 1 )) 1 0)
(if (and (= G5 2) (= S5 1 )) 1 0)
(if (and (= G6 2) (= S6 1 )) 1 0)
(if (and (= G7 2) (= S7 1 )) 1 0)
(if (and (= G8 2) (= S8 1 )) 1 0)
(if (and (= G9 2) (= S9 1 )) 1 0)
(if (and (= G10 2) (= S10 1 )) 1 0) ) 2)
;avg age white grandparents: 80
(= (/ (* (+ (* (if (and (= G1 2)
(= R1 0)) 1 0) A1)
(* (if (and (= G2 2) (= R2 0)) 1 0) A2)
(* (if (and (= G3 2) (= R3 0)) 1 0) A3)
(* (if (and (= G4 2) (= R4 0)) 1 0) A4)
(* (if (and (= G5 2) (= R5 0)) 1 0) A5)
(* (if (and (= G6 2) (= R6 0)) 1 0) A6)
(* (if (and (= G7 2) (= R7 0)) 1 0) A7)
(* (if (and (= G8 2) (= R8 0)) 1 0) A8)
(* (if (and (= G9 2) (= R9 0)) 1 0) A9)
(* (if (and (= G10 2) (= R10 0)) 1 0) A10))
10) 2) 800)
;number of childless households: 1
(xor (= (+ (if (and (= H1 1) (= G1 0)) 1 0)
(if (and (= H2 1) (= G2 0)) 1 0)
(if (and (= H3 1) (= G3 0)) 1 0)
(if (and (= H4 1) (= G4 0)) 1 0)
(if (and (= H5 1) (= G5 0)) 1 0)
(if (and (= H6 1) (= G6 0)) 1 0)
(if (and (= H7 1) (= G7 0)) 1 0)
(if (and (= H8 1) (= G8 0)) 1 0)
(if (and (= H9 1) (= G9 0)) 1 0)
(if (and (= H10 1) (= G10 0)) 1 0) ) 3)
(= (+ (if (and (= H1 2) (= G1 0)) 1 0)
(if (and (= H2 2) (= G2 0)) 1 0)
(if (and (= H3 2) (= G3 0)) 1 0)
(if (and (= H4 2) (= G4 0)) 1 0)
(if (and (= H5 2) (= G5 0)) 1 0)
(if (and (= H6 2) (= G6 0)) 1 0)
(if (and (= H7 2) (= G7 0)) 1 0)
(if (and (= H8 2) (= G8 0)) 1 0)
(if (and (= H9 2) (= G9 0)) 1 0)
(if (and (= H10 2) (= G10 0)) 1 0) ) 3))
;number of interracial married couples: 2
(= 1 (+ (if (and (= H1 1)
   (and (= G1 1) (= R1 0))) 1 0)
(if (and (= H2 1) (and (= G2 1)
   (= R2 0))) 1 0)
(if (and (= H3 1) (and (= G3 1)
   (= R3 0))) 1 0)
(if (and (= H4 1) (and (= G4 1)
   (= R4 0))) 1 0)
(if (and (= H5 1) (and (= G5 1)
   (= R5 0))) 1 0)
(if (and (= H6 1) (and (= G6 1)
   (= R6 0))) 1 0)
(if (and (= H7 1) (and (= G7 1)
   (= R7 0))) 1 0)
(if (and (= H8 1) (and (= G8 1)
   (= R8 0))) 1 0)
(if (and (= H9 1) (and (= G9 1)
   (= R9 0))) 1 0)
(if (and (= H10 1) (and (= G10 1)
   (= R10 0))) 1 0)))
(= 1 (+ (if (and (= H1 2)
   (and (= G1 1) (= R1 0))) 1 0)
(if (and (= H2 2)
   (and (= G2 1) (= R2 0))) 1 0)
(if (and (= H3 2)
   (and (= G3 1) (= R3 0))) 1 0)
(if (and (= H4 2)
   (and (= G4 1) (= R4 0))) 1 0)
(if (and (= H5 2)
   (and (= G5 1) (= R5 0))) 1 0)
(if (and (= H6 2)
   (and (= G6 1) (= R6 0))) 1 0)
(if (and (= H7 2)
   (and (= G7 1) (= R7 0))) 1 0)
(if (and (= H8 2)
   (and (= G8 1) (= R8 0))) 1 0)
(if (and (= H9 2)
   (and (= G9 1) (= R9 0))) 1 0)
(if (and (= H10 2)
   (and (= G10 1) (= R10 0))) 1 0)))
;household average ages: 62, 18
(= 620 (/ (* 10 (+
(* A1 (if (= H1 1) 1 0))
(* A2 (if (= H2 1) 1 0))
(* A3 (if (= H3 1) 1 0))
(* A4 (if (= H4 1) 1 0))
(* A5 (if (= H5 1) 1 0))
(* A6 (if (= H6 1) 1 0))
(* A7 (if (= H7 1) 1 0))
(* A8 (if (= H8 1) 1 0))
(* A9 (if (= H9 1) 1 0))
(* A10 (if (= H10 1) 1 0)))) 5))

\end{verbatim}


\end{document}




Clearly, if there is one possible solution, then suppressing
additional variables will increase the size of the solution
universe. However, the fictional statistics agency must be concerned
about more than the 

, the set of all solutions that fit the encoded set of constraints, is potentially quite large, containing many false solutions along with the true solution.
However, each time the agency publishes a new statistic, the attacker can generate new constraints and therefore narrow
down the set of possible solutions.


This system has one true
solution, equivalent to the 
ground truth, and possibly many other false solutions, which fit the
constraints but are not equivalent to the ground truth. 

Eventually, the attacker
will be able to narrow down the solution universe to just one solution, at which point the solution universe contains only
the ground truth. At this point, the attack has succeeded, and the statistical agency has completely failed to protect its respondents' privacy. Note that the cell suppression disclosure avoidance technique does not prevent the attacker from performing the reconstruction attack, but rather simply gives one fewer constraint to work with per cell suppressed.

Furthermore, even if the number of constraints is insufficient to narrow down the solution universe to just one element, the attacker can still often identify personal data for some respondents because of the high probability that all remaining solutions share values for a set of people. For example, if there are three remaining solutions in the solution universe, and all three contain a 75-year-old black male grandparent in household 1, then the attacker knows that this person is a real person in the database, even though the rest of the database may not have been reconstructed. Thus, the database has failed to protect this person's privacy even though the attack did not fully reconstruct the database.
Although public Census data tables are drawn from hundreds of millions of American individuals, the Census publishes billions of statistics from those individual responses, so there are still sufficient constraints to narrow down the solution universe enormously.


>> Need to discuss that suppressing more results increases the size of
the solution universe, but that comonalities in solutions may sill
compromise privacy.


% LocalWords:  microdata equalities


