\documentclass[jou,apacite]{apa6}
\usepackage{amsmath}

\title{Analysis of a Database Reconstruction Attack on Public Data}
\shorttitle{}

\twoauthors{Simson Garfinkel}{Christian Martindale}
\twoaffiliations{Center for Disclosure Avoidance, U.S. Census Bureau}{Center for Disclosure Avoidance, U.S. Census Bureau}

\abstract{In recent years, a certain type of malicious database attack,
the database reconstruction attack, has become increasingly
more feasible due to rapid advances in attack algorithm
sophistication. Here, we will discuss how these attacks
function, demonstrate their effectiveness and efficiency,
and provide solutions for defending against such an attack.

This paper will answer the following questions:
\begin{enumerate}
    \item What is a database reconstruction attack (DRA)?
    \item To what extent are modern DRAs effective in reconstructing the ground truth data?
    \item How scalable are DRAs with database size?
    \item How can a statistical agency guard public data against a DRA without significantly impacting the accuracy of the data?
\end{enumerate}

We will also develop a vocabulary to facilitate the discussion
of database privacy and formally define some important terms.

\textbf{Keywords: Database Reconstruction Attack, SAT solver, privacy,
disclosure avoidance}}


\begin{document}
\maketitle

\section{Problem Background}
Malicious people often seek to identify individuals from publicly
available data products such as tables and graphs.
Defending against such privacy breaches is
a high priority for statistical agencies, and so researchers
have developed a variety of techniques to prevent database users
from identifying any Personally Identifiable Information (PII)

These techniques include:
\begin{enumerate}
  \item Cell Suppression, where the values of cells with small counts or few possible
        generating combinations are removed from the published table
  \item Row Swapping, where the data rows corresponding to individuals
        with similar values in certain key cells are switched
  \item Bucketing, where numerical values are grouped into
        buckets corresponding to ranges, instead of giving the exact
        values for each entry in the table
  \item Topcoding, where the buckets at the high and low ends
        of the table are given without upper or lower bound (e.g.
        reporting the highest bucket for age as "80+" instead of
        "80-90" and "90-100")
\end{enumerate}

The goal of a Database Reconstruction Attack is to
use public data to create a mathematical system of equations,
which then can be used to reconstruct the original (before the disclosure
avoidance techniques were applied) data set.
While the above techniques are not without merit, this paper will
demonstrate that they alone are insufficient in guarding data against
a modern DRA.

\section{Vocabulary}

\begin{itemize}
\item Database Reconstruction Attack (DRA) - An attempt to determine survey response information from a publicly available data set which has had disclosure avoidance techniques applied to it.

\item Personally Identifiable Information (PII) - Data that can be used to identify a single person.

\item Constraint Equation - A mathematical equation that represents information known to be true for a set of data.

\item Solution Universe (U) - The set of all combinations of a
group of variables that satisfies a certain set of constraint equations.

\item SAT Solver - A program that uses a complex set of heuristics to find the solution universe given a set of constraint equations.

\item Noise Addition - The addition of small random numbers to statistical table values before publication for the purpose of protecting respondent privacy.
\end{itemize}

\section{The Database Reconstruction Attack: An Example}
To understand how a DRA works, let us consider a simple example data table generated from the following survey given to two households:
\begin{verbatim}
Please provide the following for each member
of your family:
Age, Sex, Race, Generation.
\end{verbatim}
Tables 1 and 2 summarize the survey responses for the two households surveyed.
\begin{table}[!htb]
\caption{Household 1}\label{tab1}
\begin{tabular}{cccc}
\hline\\[-1.5ex]
Age & Sex & Race & Gen \\[0.5ex]
\hline\\[-1.5ex]
55 & Female & White & Grandparent\\[0.5ex]
24 & Male & White & Parent\\[0.5ex]
28 & Female & Black & Parent\\[0.5ex]
5 & Female & Black & Child\\[0.5ex]
8 & Male & White & Child\\[0.5ex]
\hline
\end{tabular}
\end{table}

\begin{table}[!htb]
\caption{Household 2}\label{tab2}
\begin{tabular}{cccc}
\hline\\[-1.5ex]
Age & Sex & Race & Gen \\[0.5ex]
\hline\\[-1.5ex]
50 & Male & White & Parent\\[0.5ex]
45 & Male & White & Parent\\[0.5ex]
15 & Female & White & Child\\[0.5ex]
17 & Female & White & Child\\[0.5ex]
\hline
\end{tabular}
\end{table}

The statistical agency then publishes the following data based
on the survey results: \footnote {Note that statistics formed from groups made up of only one individual are redacted with "XXXX". This is an example of cell suppression used to protect the privacy of those individuals.}

\begin{table}[!htb]
\caption{Survey Results}\label{tab3}
\begin{tabular}{ccc}
\hline\\[-1.5ex]
Group & Number & Average Age \\[0.5ex]
\hline\\[-1.5ex]
Individuals & 9 & 27.44 \\[0.5ex]
Males & 4 & 31.75 \\[0.5ex]
Females & 5 & 24 \\[0.5ex]
Parents & 4 & 36.75 \\[0.5ex]
Grandparents & 1 & XXXX \\[0.5ex]
Children (0-12) & 2 & 6.5 \\[0.5ex]
Children (13-17) & 2 & 16 \\[0.5ex]
Whites & 7 & 30.57 \\[0.5ex]
Blacks & 2 & 24 \\[0.5ex]
Black parents & 1 & XXXX \\[0.5ex]
Black children & 1 & XXXX\\[0.5ex]
White parents & 3 & 39.66 \\[0.5ex]
White children & 3 & 13.33 \\[0.5ex]
Households & 2 & 27.85 \\[0.5ex]
\hline
\end{tabular}
\end{table}

Retabulating these results in a numerical format with the keys in Table 4 gives the results in Table 5.

\begin{table}[!htb]
\caption{Survey Results}\label{tab4}
\begin{tabular}{c|c}
\hline\\[-1.5ex]
Key & Value \\[0.5ex]
\hline\\[-1.5ex]
Male & 0 \\[0.5ex]
Female & 1 \\[0.5ex]
White & 0 \\[0.5ex]
Black & 1 \\[0.5ex]
Child & 0 \\[0.5ex]
Parent & 1 \\[0.5ex]
Grandparent & 2 \\[0.5ex]
\hline
\end{tabular}
\end{table}

\begin{table}[!htb]
\caption{Survey Results}\label{tab5}
\begin{tabular}{c|c|c|c|c|c}
\hline\\[-1.5ex]
ID & Household & Age & Sex & Race & Generation \\[0.5ex]
\hline\\[-1.5ex]
1 & 1 & 24 & 0 & 0 & 1  \\[0.5ex]
2 & 1 & 28 & 1 & 1 & 1  \\[0.5ex]
3 & 1 & 55 & 1 & 0 & 2  \\[0.5ex]
4 & 1 & 5 & 1 & 1 & 0  \\[0.5ex]
5 & 1 & 8 & 0 & 0 & 0  \\[0.5ex]
6 & 2 & 50 & 0 & 0 & 1  \\[0.5ex]
7 & 2 & 45 & 1 & 0 & 0  \\[0.5ex]
8 & 2 & 15 & 1 & 0 & 0  \\[0.5ex]
9 & 2 & 17 & 1 & 0 & 0 \\[0.5ex]
\hline
\end{tabular}
\end{table}


The data in Table 5 is the 'ground truth' database that the attacker wishes to reconstruct. At the start of the attack, the attacker writes the following table of 45 unknowns:

\begin{table}[!htb]
\caption{Survey Results}\label{tab6}
\begin{tabular}{c|c|c|c|c|c}
\hline\\[-1.5ex]
ID & Household & Age & Sex & Race & Generation \\[0.5ex]
\hline\\[-1.5ex]
1 & H1 & A1 & S1 & R1 & G1  \\[0.5ex]
2 & H2 & A2 & S2 & R2 & G2  \\[0.5ex]
3 & H3 & A3 & S3 & R3 & G3  \\[0.5ex]
4 & H4 & A4 & S4 & R4 & G4  \\[0.5ex]
5 & H5 & A5 & S5 & R5 & G5  \\[0.5ex]
6 & H6 & A6 & S6 & R6 & G6  \\[0.5ex]
7 & H7 & A7 & S7 & R7 & G7  \\[0.5ex]
8 & H8 & A8 & S8 & R8 & G8  \\[0.5ex]
9 & H9 & A9 & S9 & R9 & G9  \\[0.5ex]
\hline
\end{tabular}
\end{table}

The reconstruction attack works by identifying constraint
equations given by the data table. For example, we have the
following statistic:
\begin{verbatim}
Individuals: 9, 27.44
\end{verbatim}

This can be written as a linear constraint equation: \footnote{In the actual calculation, mathematical transformations must be applied to avoid floating point rounding errors, but that is beyond the scope of this paper.}
\[(A1 + A2 + A3 + A4 + A5 + A6 + A7 + A8 + A9)*100/9 = 27.44\]

This equation is in 9 unknowns, and so is unsolvable alone.
However, the attacker can write more constraint equations.

\begin{verbatim}
Grandparents: 1, XXX
\end{verbatim}


Becomes: \footnote{X==Y evaluates to 0 if X!=Y and 1 if $X=Y$.}
\begin{align*}
& (G1==2) + (G2==2) + (G3==2) + (G4==2) + \\
& (G5==2)+ (G6==2) + (G7==2) +\\
& (G8==2) + (G9==2) = 1
\end{align*}

\begin{verbatim}
Children 0-12: 2, 6.5
\end{verbatim}

Becomes the following two equations:
\begin{align*}
& (G1==0) + (G2==0) + (G3==0)+\\
& (G4==0)+  (G5==0) + (G6==0) +\\
& (G7==0) + (G8==0) + G9==0) = 2
\end{align*}

\begin{align*}
& (A1 * (G1==0) + A2 * (G2==0) + A3 * (G3==0) +\\
& A4 * (G4==0) +  A5 * (G5==0) + A6 * (G6==0) + \\
& A7 * (G7==0) + A8 * (G8==0) + A9 * (G9==0)) = 13
\end{align*}
Once the attacker has converted all the published statistics
into equations like the ones above, he will have a system
of a large number of equations in 45 unknowns. This system has one 'true' solution, $S_g$, equivalent to the
ground truth, and possibly many other 'false' solutions, $[S_1...S_n]$.
The Solution Universe, U, is potentially originally quite large, with $U = [S_1...S_n]$

However, each time the agency publishes a new statistic, the attacker
can generate new constraints and therefore narrow
down the set of possible solutions. Eventually, the attacker
will obtain $U = [S_g]$, at which point he has
found the ground truth and reconstructed the database
successfully. Note here that the cell suppression disclosure avoidance technique does not prevent the attacker from performing the reconstruction attack, but rather simply gives him one fewer constraint to work with per cell suppressed.
Even if the number of constraints is insufficient to narrow down U

\section{Methods of Attack}

\section{Revisiting the Example}

\section{Defending Against a DRA}


\section{Appendix}

\subsection{SAT Solvers}

\subsection{Scalability: The Zebra Problem}

\subsection{How Noise Addition Combats a DRA}


Results are presented in Table~\ref{tab1}.
\begin{table}[!htb]
\caption{Sample table.}\label{tab1}
\begin{tabular}{ccc}
\hline\\[-1.5ex]
AAA & BBB & CCC \\[0.5ex]
\hline\\[-1.5ex]
1.0 & 2.0 & 3.0\\[0.5ex]
1.0 & 2.0 & 3.0\\[0.5ex]
\hline
\end{tabular}
\end{table}


\end{document}
