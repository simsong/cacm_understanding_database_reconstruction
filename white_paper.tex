\newif\ifanonymized
\anonymizedtrue
%\anonymizedfalse
\documentclass[runningheads]{llncs}
%cmap has to be loaded before any font package (such as cfr-lm)
\usepackage{cmap}
\usepackage[T1]{fontenc}
\usepackage{graphicx}
\usepackage[ngerman,english]{babel}
\usepackage{verbatim}
%better font, similar to the default springer font
%cfr-lm is preferred over lmodern. Reasoning at http://tex.stackexchange.com/a/247543/9075
\usepackage[%
rm={oldstyle=false,proportional=true},%
sf={oldstyle=false,proportional=true},%
tt={oldstyle=false,proportional=true,variable=true},%
qt=false%
]{cfr-lm}
%Sorts the citations in the brackets
%It also allows \cite{refa, refb}. Otherwise, the document does not compile.
%  Error message: "White space in argument"
\usepackage{cite}
%extended enumerate, such as \begin{compactenum}
\usepackage{paralist}
%for easy quotations: \enquote{text}
\usepackage{csquotes}

%enable margin kerning
\usepackage{microtype}

%tweak \url{...}
\usepackage{url}
%improve wrapping of URLs - hint by http://tex.stackexchange.com/a/10419/9075
\makeatletter
\g@addto@macro{\UrlBreaks}{\UrlOrds}
\makeatother
%required for pdfcomment later
\usepackage{xcolor}
%enable nice comments
%this also loads hyperref
\usepackage{pdfcomment}
%enable hyperref without colors and without bookmarks
\hypersetup{hidelinks,
   colorlinks=true,
   allcolors=black,
   pdfstartview=Fit,
   breaklinks=true}
%enables correct jumping to figures when referencing
\usepackage[all]{hypcap}

\newcommand{\commentontext}[2]{\colorbox{yellow!60}{#1}\pdfcomment[color={0.234 0.867 0.211},hoffset=-6pt,voffset=10pt,opacity=0.5]{#2}}
\newcommand{\commentatside}[1]{\pdfcomment[color={0.045 0.278 0.643},icon=Note]{#1}}

%compatibality with packages todo, easy-todo, todonotes
\newcommand{\todo}[1]{\commentatside{#1}}
%compatiblity with package fixmetodonotes
\newcommand{\TODO}[1]{\commentatside{#1}}

%enable \cref{...} and \Cref{...} instead of \ref: Type of reference included in the link
\usepackage[capitalise,nameinlink]{cleveref}
%Nice formats for \cref
\crefname{section}{Sect.}{Sect.}
\Crefname{section}{Section}{Sections}

\usepackage{xspace}
%\newcommand{\eg}{e.\,g.\xspace}
%\newcommand{\ie}{i.\,e.\xspace}
\newcommand{\eg}{e.\,g.,\ }
\newcommand{\ie}{i.\,e.,\ }

%introduce \powerset - hint by http://matheplanet.com/matheplanet/nuke/html/viewtopic.php?topic=136492&post_id=997377
\DeclareFontFamily{U}{MnSymbolC}{}
\DeclareSymbolFont{MnSyC}{U}{MnSymbolC}{m}{n}
\DeclareFontShape{U}{MnSymbolC}{m}{n}{
    <-6>  MnSymbolC5
   <6-7>  MnSymbolC6
   <7-8>  MnSymbolC7
   <8-9>  MnSymbolC8
   <9-10> MnSymbolC9
  <10-12> MnSymbolC10
  <12->   MnSymbolC12%
}{}
\DeclareMathSymbol{\powerset}{\mathord}{MnSyC}{180}

% correct bad hyphenation here
\hyphenation{op-tical net-works semi-conduc-tor}

%% END COPYING HERE

\usepackage{amsmath}
\usepackage{amssymb}
\usepackage{siunitx}


\begin{document}
\title{Database Reconstruction Attack on Public Data}
\titlerunning{Database Reconstruction}
\ifanonymized
\author{Anonymized Author(s)}
\institute{Institute for Anonymous Papers}
\else
\author{Christian Martindale \and Simson Garfinkel}
\institute{Center for Disclosure Avoidance, U.S. Census Bureau}
\fi

\maketitle
\begin{abstract}
Statistical agencies are mandated to publish summary statistics and
micro data while not providing data users with the ability to derive
specific attributes for particular individuals or
establishments. Today these privacy guarantees are assured through the
practical of \emph{Statistical Disclosure Limitation} (SDL)
techniques, such as suppressing cells and generalizing categories in
summary tables. Although these techniques are not be sufficient to
prevent a database reconstruction attack of the sort anticipated by
Dinur and Nissim\cite{noise}, statistical agencies have been slow to
adopt disclosure limitation techniques based on formal privacy
techniques such as differential privacy---presumably because there is
little knowledge of formal privacy techniques within the broader
official statistics community.  This paper discusses how a database
reconstruction attack functions and demonstrate their effectiveness
and efficiency on a toy example. We then show how SDL techniques based
on additive noise can be effective at defending against database
reconstruction.
\end{abstract}

\begin{keywords}
database reconstruction attack, SAT solver, privacy, disclosure avoidance
\end{keywords}


\section{Introduction}
Working Paper \#22 of the U.S. Federal Committee on Statistical
Methodology\cite{workingpaper22} outlines the currently accepted best
practice for statistical agencies to follow when they prepare and
release both statistical data and ``de-identified''
microdata. Broadly, statistical agencies are charged with releasing
high-quality data to further public policy goals, but they are
prohibited from releasing data that might result in the identification
of a data about a specific individual or establishment, or the linkage
of microdata to an identity.

Working Paper \#22 outlines a number of approaches that statistical
agencies can use for protecting respondent data. This techniques include:
\begin{enumerate}
  \item \textbf{Cell Suppression}, where the values of cells with small counts or few possible
        generating combinations are removed from the published table
  \item \textbf{Row Swapping}, where the data rows corresponding to individuals
        with similar values in certain key cells are switched
  \item \textbf{Generalization}, where numerical values are grouped into
        buckets corresponding to ranges, instead of giving the exact
        values for each entry in the table
  \item \textbf{Top-and-bottom-coding}, where the statistical groups at the high and low ends
        of the table are given without upper or lower bound (e.g.
        reporting the highest group for age as 80+ instead of
        80-90 and 90-100)
\end{enumerate}

While it makes intuitive sense that these techniques hamper the
ability of a  data intruder\cite{data-intruder} to recover respondent data from the
statistical release, such hunches do not constitute rigorous
mathematical proofs. Absent a formal definition of privacy and
mathematical proofs showing that a specific disclosure limitation
technique realizes that definition, there is no way to know if 
technique actually protect privacy, or if that is merely wishful thinking.

In this paper, we use
the term \emph{database reconstruction attack} (DRA) to describe the process of
taking a published set of statistical tables and deriving the
underlying sensitive data. Such attacks are possible and surprisingly
practical, and this fact has been known for more than 15
years\cite{noise}. Yet surprisingly, today most of
statistical agencies still rely on the disclosure
limitation techniques described in Working Paper \#22, rather than
having adopted new techniques based on differential
privacy\cite{Dwork:2006:CNS:2180286.2180305}. 

The contribution of this paper, then, is to present a model example of how
a database reconstruction attack might be implemented against a data
release from an official statistics agency, and then to show how a
formally private technique can protect against such an
attack. Although the possibility of database reconstruction is
presented in Dinur and Nissum's original paper\cite{noise}, we are not
aware of any end-to-end example showing the threat, a worked attack,
and the results of modern defenses. 

\section{Related Work}

Dinur and Nissim\cite{noise} showed that the underlying
confidential data of a statistical database can be reconstructed with
a surprisingly small number of queries. In practice, most statistical
agencies perform these queries themselves when they release
statistical tables. Thus, Dinur and Nissim's primary finding 
is that a statistical agency that publishes too many statistical
tables drawn from the same confidential dataset risks inadvertently
compromising that dataset unless it takes specific protective measures.

Statistical tables create the possibility of database reconstruction
because they form a set of constraints for which there is ultimately
only one exact solution. Restricting
the number or specific types of queries---for example, by suppressing
results from a small number of respondents---is often insufficient to prevent access
to indirectly identifying information, becasue the system's refusal to
answer a ``dangerous'' query itself gives the user information. The
authors found that if a database is modeled as a string of $n$ bits,
then at least $\sqrt{n}$ bits must be modified by adding noise to
protect individuals from being identified.


Kasiviswanathan, Rudelson, and Smith\cite{Kasiviswanathan:2013:PLR:2627817.2627919} introduced
the concept of the linear reconstruction attack which is the root
behind the generic DRA. The key concept is that,
given nonsensitive data such as zip code or gender, an attacker
can construct a matrix of linear equalities that can often be solved
in polynomial time. The paper also analyzes a common reconstruction
technique known as least squares decoding, where the attacker sets up
a goal function to minimize the square of the distance between two
databases in order to reconstruct the original database. 

Brown and Heathers\cite{doi:10.1177/1948550616673876} developed the
granularity-related inconsistency of means (GRIM) test in response to
observed inconsistencies in published data from psychological
journals. This test is centered around the premise that, for
statistics drawn from integer data, only certain means are
possible. The GRIM test determines whether reported means could
possibly have come from data sets with a certain size, granularity,
and group number. In surveying 71 published articles, the authors
found 36 papers with one inconsistency and 16 with two or more
inconsistencies. Although this test was intended to detect possible
errors or mean falsification in published articles, the concept of
drawing inferential conclusions about a data set based only off of
published statistics is a key concept behind the DRA.

\section{The Database Reconstruction Attack: An Example}

To present a DRA, we consider a hypothetical
census of a fictional block conducted by a fictional statistical
agency. For every household, the agency collects each member's age,
sex, race, and their generation in the household.  Because the
confidential data will not be released in order to protect respondent privacy, the statistical agency 
publishes every conceivable statistic that might be of use to a
potential demographers, sociologists and economists. Based on user
feedback from previous census reports, the agency has come up with the
a set of statistics, which we present here as Table~\ref{publishedstatsbig}.

Note that in order to protect respondent privacy, the agency has
suppressed all statistics that are based on a single
individual. 

\begin{table}
\begin{center}
\begin{tabular}{l|l|c|c}
Item & Group & Number & Average Age \\
\hline
1A & Individuals & 10 & 40 \\
1B & Males & 5 & 34 \\
1C & Females & 5 & 46 \\
1D & Whites & 5 & 50 \\
1E & Blacks & 5 & 30 \\
\hline
2A & Children (0-12) & 3 & 10 \\
2D & White children & 1 & \multicolumn{1}{c}{\rule{6mm}{3mm}} \\
2E & Black children & 2 & 10 \\
\hline
3A & Parents & 4 & 32.5 \\
3B & Male parents & 2 & 30 \\
3C & Female parents & 2 & 35 \\
35 & Parents over 40 & 0 & -- \\
\hline
4A & Grandparents & 3 & 80 \\
4B & Male grandparents & 1 & \multicolumn{1}{c}{\rule{6mm}{3mm}} \\
4C & Female grandparents & 2 & 75 \\
4D & White grandparents & 2 & 80 \\
4E & Black grandparents & 1 & \multicolumn{1}{c}{\rule{6mm}{3mm}} \\
\hline
5A & Households & 2 & 40 \\
5B & Tri-generational households & 0 & -- \\
5C & Single-parent households & 0 & -- \\
5D & Childless households & 1 & \multicolumn{1}{c}{\rule{6mm}{3mm}} \\
5E & Interracial married couples & 2 & 32.5 \\
5F & Same-sex married couples & 0 & -- \\
5G & Households $\geq 40\% $ female & 2 & 40 \\
5H & Households $\geq 40\% $ black & 2 & 40 \\
\hline
\end{tabular}
\end{center}
\caption{Fictional statistical data published by the fictional
  statistics agency. Item numbers are for identification purpose
  only. Note that statistics 2D, 4B, 4E and 5D have been suppressed as part
  of the statistical disclosure limitation process.}\label{publishedstatsbig}
\end{table}

The goal of the DRA is to reconstruct the number of households and,
for each household, to learn the age, sex, race, and generation of
each member. To do this, the data intruder views these attributes as a set of five
unknown variables. Since there are 10 individuals in
the fictitious census, there are 50 unknowns (Table~\ref{50unknowns}).

\begin{table}
\begin{minipage}[t]{3in}
\begin{tabular}{cccccc}
ID & Household & Age & Sex & Race & Generation \\
\hline
\hline
1 & H1 & A1 & S1 & R1 & G1  \\
\hline
2 & H2 & A2 & S2 & R2 & G2  \\
\hline
3 & H3 & A3 & S3 & R3 & G3  \\
\hline
4 & H4 & A4 & S4 & R4 & G4  \\
\hline
5 & H5 & A5 & S5 & R5 & G5  \\
\hline
6 & H6 & A6 & S6 & R6 & G6  \\
\hline
7 & H7 & A7 & S7 & R7 & G7  \\
\hline
8 & H8 & A8 & S8 & R8 & G8  \\
\hline
9 & H9 & A9 & S9 & R9 & G9  \\
\hline
10 & H10 & A10 & S10 & R10 & G10  \\
\hline
\end{tabular}
\end{minipage}
\begin{minipage}[t]{1in}
\begin{tabular}{c|c}
Key & Value \\
\hline
Male & 0 \\
Female & 1 \\
\hline
White & 0 \\
Black & 1 \\
\hline
Child & 0 \\
Parent & 1 \\
Grandparent & 2 \\
\hline
\end{tabular}
\end{minipage}
\caption{Unknowns for the 10 fictitious individuals whose statistical
  data are presented in Table~\ref{publishedstatsbig}. Unknowns
  H\textit{n} are the household, A\textit{n} the age, S\textit{n} the
  sex, R\textit{n} the race and G\textit{n} the generation. Sex, Race
  and Generation categorical values; they are converted into numerical
  values using the key at the right.}\label{50unknowns}
\end{table}

The data intruder proceeds by examining the published tables and
identifying how they can be translated into constraints on the private
data. The constraints are expressed as mathematical formulae
representing rules that the ground truth must satisfy. Below we
identify how specific released statistics can be mapped to
constraints.

We can encode the constraint based on statistic 1A as a linear
constraint equation with 10 unknowns, each of which has the range
$0\ldots120$, using the symbols for age $A1...A10$
(Table~\ref{50unknowns}), where $=$ is the constraint operator
establishing that the two sides must be equal:

\begin{equation}
\frac{A1 + A2 + A3 + A4 +...+ A10}{10} = 40
\end{equation}\label{eq1}

Two constraints can be encoded based on statistic 1B: the existence of
five men, and the average age of 34. To encode these kinds of
conditional constraints, we must introduce the == operator,
which here is a test for equivalence. This operator returns 1 if the
left and right hand sides are equal, and 0 otherwise. 
Five men is thus:

\begin{equation}
\begin{split}
(S1==0) + (S2==0) + (S3==0) + (S4==0) + (S5==0) +  & \\
(S6==0) + (S7==0) + (S8==0) + (S9==0) + (S10==0) & = 3 
\end{split}
\end{equation}

And five men with the average age of 34 is:

\begin{equation}
\begin{split}
A1\times(S1==0) + A2\times(S2==0) + A3\times(S3==0) + &\\
A4\times(S4==0) + A5\times(S5==0) + A6\times(S6==0) + &\\
A7\times(S7==0) + A8\times(S8==0) + A9\times(S9==0) + &\\
                                    A10\times(S10==0) & = 34\times5
\end{split}
\end{equation}

Once the attacker has converted all the published statistics
into equations like the ones above, the equations together form a
system of many equations in 50 unknowns. 
There is a \textit{solution universe} of all possible solutions to
this set of simultaneous equations. If there is a single possible
solution, then the published statistics completely reveal the
underlying confidential data---the \emph{ground truth}. However, if
there is more than one possible solution, then one of those solutions
is correct, and the others are not. If the equations have no solution,
then the set of published statistics is inconsistent.

\section{Performing the Attack}
A straightforward but possibly inefficient approach to solving this
system of simultaneous equations is to perform a brute force
search---that is, to try all possible values of all possible variables
and record the ones that work.

Ordinarily a brute force search on even this toy problem would be
unreasonable: there are 10 age variables that can range from 0...120,
30 binary variables (household, sex and race), and ten trinary
variable. All together, there are $120^{10} \times 2^{30} \times
3^{10} \approx 4 \times 10^{34}$ possible combinations. Even if we
could try a billion a second, trying all possible combinations would
take a billion billion years.

Those familiar with complexity theory will have recognized by now that
the system of variables can be expressed as a satisfiability (SAT)
problem purely Boolean variables. This is done by encoding each of the
integer variables and trinary variables as themselves being derived
from a set of Boolean variables with still more constraints. Encoding
the DRA has a SAT problem allows it to be solved using a SAT Solver
programs that have been developed over the past two
decades that use sophisticated heuristics that can rapidly solve many
SAT problems. 

Those not familiar with modern SAT solvers may be somewhat
incredulous at the suggestion of using attempting to use them to solve
a problem with a work factor of roughly $10^{34}\approx2^{113}$.
However, ``The past few years have seen an enormous progress in the performance
of Boolean satisfiability (SAT) solvers. Despite the worst-case
exponential run time of all known algorithms, satisfiability solvers
are increasingly leaving their mark as a general-purpose tool in areas
as diverse as software and hardware verification,
automatic test pattern generation, planning,
scheduling, and even challenging problems from algebra. Annual SAT
competitions have led to the development of dozens 
of clever implementations of such solvers, an exploration of many new
techniques, and the creation of an extensive suite of real-world
instances as well as challenging hand-crafted benchmark
problems. Modern SAT solvers provide a ``black-box'' procedure that
can often solve hard structured problems with over a million variables and
several million constraints.''\cite[references omitted]{Gomes200889}.

Many SAT solvers take their input in the so-called DIMACS file format,
which specifies a single equation in conjunctive normal form. Because
the DIMACS format can be difficult to understand, we Sugar\cite{sugar} to translate our system of
constraints into the DIMACS format. Sugar accepts input as a series of
\textit{s-expressions}\cite{McCarthy:1960:RFS:367177.367199}. For example, the
constraint \ref{eq1} can be encoded as the following s-expression:
\begin{verbatim}
(= (/ (+ A1 A2 A3 A4 A5 A6 A7 A8 A9 A10) 
         10) 
      40)
\end{verbatim}

We created s-expressions for the constraints in Table~\ref{publishedstatsbig}.

Output is given in Table~\ref{sugarbig} and is tabulated for readability. Although the ID numbers are permutated, the ten people in the SAT solver output are identical to the ten people from the ground truth in Table~\ref{resultsbig}, so the attack has succeeded in reconstructing the entire database and narrowing down the solution universe to size 1. Finding this solution required 35 seconds on a single 2012 MacBook Pro with an Intel i7 2.9GHz processor.

At past international SAT solver competitions, SAT solvers such as PicoSAT have solved problems with tens of millions of variables in less than 20 minutes \cite{satcomp}, demonstrating that dataset size and constraint number is not an insurmountable obstacle to this type of DRA.

\begin{table}
\begin{tabular}{rllp{1in}||rllp{1in}}
\multicolumn{4}{c||}{Household \#1}    & \multicolumn{4}{c}{Household \#2} \\
Age & Sex & Race & Gen                 & Age & Sex & Race & Gen \\
\hline
90 & Male & White & Grandparent        & 40 & Female & White & Parent\\
80 & Female & Black & Grandparent      & 20 & Male & Black & Parent\\
70 & Female & White & Grandparent      & 10 & Female & White & Child\\
40 & Male & White & Parent             & 10 & Male & Black & Child\\
30 & Female & Black & Parent           & 10 & Male & Black & Child\\
\hline
\end{tabular}
\caption{Responses from a two fictional households
for a survey that collects Age, Sex, Race and Generation of each resident. This is
the unpublished, confidential data collected by a fictional statistical
agency.}\label{responses}
\end{table}


\begin{table}
\begin{tabular}{c|c|c|c|c|c}
ID & Household & Age & Sex & Race & Generation \\
\hline
1 & 1 & 80 & 1 & 1 & 2  \\
2 & 1 & 40 & 0 & 0 & 1  \\
3 & 1 & 70 & 1 & 0 & 2  \\
4 & 1 & 30 & 1 & 1 & 1  \\
5 & 1 & 90 & 0 & 0 & 2  \\
6 & 2 & 10 & 0 & 1 & 0  \\
7 & 2 & 10 & 0 & 1 & 0  \\
8 & 2 & 10 & 1 & 0 & 0  \\
9 & 2 & 40 & 1 & 0 & 1 \\
10 & 2 & 20 & 0 & 1 & 1 \\
\hline
\end{tabular}
\caption{Survey results}
\label{resultsbig}
\end{table}



\begin{table}
\begin{tabular}{c|c|c|c|c|c}
ID & Household & Age & Sex & Race & Generation \\
\hline
1 & 2 & 10 & 1 & 0 & 0  \\
2 & 2 & 20 & 0 & 1 & 1  \\
3 & 1 & 30 & 1 & 1 & 1  \\
4 & 2 & 10 & 0 & 1 & 0  \\
5 & 2 & 40 & 1 & 0 & 1  \\
6 & 2 & 10 & 0 & 1 & 0  \\
7 & 1 & 40 & 0 & 0 & 1  \\
8 & 1 & 80 & 1 & 1 & 2  \\
9 & 1 & 90 & 0 & 0 & 2 \\
10 & 1 & 70 & 1 & 0 & 2 \\
\hline
\end{tabular}
\caption{Sugar output when run on the encoded statistics in Appendix 3}\label{sugarbig}
\end{table}


\section{Defending Against a DRA}
As demonstrated above, new techniques must be developed in
order to protect databases against reconstruction attacks, as traditional techniques such as cell suppression are insufficient defense. One of the simplest and most effective techniques in defending against the DRA is noise infusion, where the publishing agency adds random values to data before publication in order to increase the size of the solution universe. For example, if the statistical agency takes a true value of age = 10, then randomly adds either -2, -1, 0, 1, or 2, and then calculates and publishes statistics consistent with this new age, there are now five elements in the solution universe for this value, where without noise infusion there would be only one. If two responses are processed with noise of this sort, there are now $5^2 = 25$ possible solutions. If 10 people undergo noise infusion of this type, there are $5^{10}$ --- nearly 10 million --- possible solutions. Notice that the statistical agency can reveal that noise infusion has taken place, and can even reveal how much noise has been added, without compromising the effectiveness of said noise infusion. Noise infusion allows an agency to publish more statistics without fearing that it has given attackers too many constraints to work with.

In the following example of noise infusion, we use noise generated
from a distribution called the Laplace distribution
\cite{Dwork:2006:CNS:2180286.2180305}. We choose this distribution
because, when correctly applied, this type of noise addition gives a
strong mathematical guarantee of individual privacy. To demonstrate
how effective adding Laplace noise is against the DRA, we reconsider
the example from the previous section. This time, the statistical
agency applies the Laplace transformation with mean 0 and exponential
decay 3 to the survey responses to the ages before publishing any
statistical products. The new published data product, analogous to
Table~\ref{publishedstatsbig}, is shown in
Table~\ref{publishedstatsnoise}.

\begin{table}[t]
\begin{tabular}{c|c|c}
Group & Number & Average Age \\
\hline
Individuals & 10 & 38.57 \\
Males & 5 & 33.21 \\
Females & 5 & 43.92 \\
Whites & 5 & 47.82 \\
Blacks & 5 & 29.31 \\
\hline
Children (0-12) & 3 & 8.50 \\
White children & 1 & \multicolumn{1}{c}{\rule{6mm}{3mm}} \\
Black children & 2 & 8.75 \\
\hline
Parents & 4 & 31.41 \\
Male parents & 2 & 29.77 \\
Female parents & 2 & 33.06 \\
Parents over 40 & 0 & -- \\
\hline
Grandparents & 3 & 78.18 \\
White grandparents & 2 & 77.64 \\
Black grandparents & 1 & \multicolumn{1}{c}{\rule{6mm}{3mm}} \\
Male grandparents & 1 & \multicolumn{1}{c}{\rule{6mm}{3mm}} \\
Female grandparents & 2 & 72.76 \\
\hline
Households & 2 & 38.57 \\
Tri-generational households & 0 & -- \\
Single-parent households & 0 & -- \\
Childless households & 1 & \multicolumn{1}{c}{\rule{6mm}{3mm}} \\
Interracial married couples & 2 & 31.41 \\
Same-sex married couples & 0 & -- \\
Households $\geq 40\% $ female & 2 & 38.57 \\
Households $\geq 40\% $ black & 2 & 38.57 \\

\hline
\end{tabular}
\caption{Data publication with added Laplace noise}\label{publishedstatsnoise}
\end{table}
It is important to note that the infusion of the noise did not significantly impact the reported statistics, thus preserving their utility to nonintrusive users. However, when the attacker attempts to perform the DRA using the same process as before, Sugar returns "Unsatisfiable". The noise infusion has defeated the DRA, and the statistical agency has upheld its legal obligation to protect respondent privacy.

\section{Problem Analysis}

With the dramatic improvement in the efficiency of SAT solvers in the last decade, the database reconstruction attack is no longer a solely theoretical danger. The vast amount of data products the Census publishes each year gives a determined and informed attacker more than enough constraints to reconstruct some or all of a target database and breach the privacy of millions of people. Although the Census Bureau's current disclosure avoidance technique suite is often sufficient to defend against a cursory attempt at an identification attack, we have demonstrated through attack simulations that cell suppression and grouping are not completely against the power of more sophisticated forms of attack.

The most effective technique for defending against a DRA is the infusion of noise to results before publication. If the noise is generated by correctly applying the Laplace mechanism, it is possible to guarantee individuals' choice to respond to a survey cannot harm them while simultaneously preserving the accuracy of the data for benign use. In infusion to continuing to use the traditional disclosure avoidance techniques, data agencies should make probabilistic noise infusion a required part of the pre-publication disclosure review process in order to protect respondent privacy.

\section{References}

\bibliographystyle{splncs03}
\bibliography{white_paper}


\section{Appendices}

\subsection{SAT and SAT Solvers}

The Boolean satisfiability problem, known as SAT, was the first problem to be proven NP-complete \cite{cooklevin}. This problem asks, for a given Boolean formula, whether replacing each variable with either True or False can make the formula evaluate to True.   A consequence of SAT being NP-complete is that any problem can be reduced in polynomial time (i.e., quickly) to an instance of the SAT problem. Once a problem has been reduced to an instance of the SAT problem, this SAT problem can be fed into a SAT solver to find possible solutions. Although the SAT problem is not solvable by algorithms in polynomial time, researchers have found many heuristic techniques for expediting this process. SAT solvers combine a variety of these techniques into one complex process, resulting in polynomial time solutions for the SAT problem in many cases.

Modern SAT solvers use a heuristic technique called Conflict-Driven Clause Learning, commonly referred to as CDCL. CDCL works by the following process \cite{cdcl}:

\begin{enumerate}

\item Assign a value to a variable arbitrarily.
\item Use this assignment to determine values for the other variables in the formula (a process known as unit propagation).
\item If a conflict is found, backtrack to the clause that made the conflict occur and undo variable assignments made after that point.
\item Add the negation of the conflict-causing clause as a new clause to the master formula and resume from step 1.

\end{enumerate}

This process is much faster at solving SAT problems than previous processes used in SAT solvers because adding conflicts as new clauses has the potential to avoid wasteful 'repeated backtracks'. Additionally, CDCL and its predecessor algorithm, DPLL, are both provably complete algorithms and will always return either a solution or "Unsatisfiable" if given enough time and memory.

There are a wide variety of SAT solvers available to the public for minimal or no cost. Although a SAT solver requires the user to translate the problem into Boolean formulae before use,  programs such as Naoyuki Tamura's Sugar facilitate this process by translating user-input mathematical and English constraints into Boolean formulae automatically.



\subsection{Sugar Input}

Sugar input is given in a standard Constraint Satisfaction Problem (CSP) file format. A constraint must be given on a single line of the file, but here we separate most constraints into multiple lines for readability. Constraint equations are separated by comments describing what statistics they encode.

Input for the mock survey from Section 4 is as follows:

\verbatiminput{constraints.csp}


\end{document}




Clearly, if there is one possible solution, then suppressing
additional variables will increase the size of the solution
universe. However, the fictional statistics agency must be concerned
about more than the 

, the set of all solutions that fit the encoded set of constraints, is potentially quite large, containing many false solutions along with the true solution.
However, each time the agency publishes a new statistic, the attacker can generate new constraints and therefore narrow
down the set of possible solutions.


This system has one true
solution, equivalent to the 
ground truth, and possibly many other false solutions, which fit the
constraints but are not equivalent to the ground truth. 

Eventually, the attacker
will be able to narrow down the solution universe to just one solution, at which point the solution universe contains only
the ground truth. At this point, the attack has succeeded, and the statistical agency has completely failed to protect its respondents' privacy. Note that the cell suppression disclosure avoidance technique does not prevent the attacker from performing the reconstruction attack, but rather simply gives one fewer constraint to work with per cell suppressed.

Furthermore, even if the number of constraints is insufficient to narrow down the solution universe to just one element, the attacker can still often identify personal data for some respondents because of the high probability that all remaining solutions share values for a set of people. For example, if there are three remaining solutions in the solution universe, and all three contain a 75-year-old black male grandparent in household 1, then the attacker knows that this person is a real person in the database, even though the rest of the database may not have been reconstructed. Thus, the database has failed to protect this person's privacy even though the attack did not fully reconstruct the database.
Although public Census data tables are drawn from hundreds of millions of American individuals, the Census publishes billions of statistics from those individual responses, so there are still sufficient constraints to narrow down the solution universe enormously.


>> Need to discuss that suppressing more results increases the size of
the solution universe, but that comonalities in solutions may sill
compromise privacy.


% LocalWords:  microdata equalities satisfiability


